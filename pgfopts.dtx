% \iffalse meta-comment
%
% Copyright (C) 2008 by
%    Joseph Wright <joseph.wright@morningstar2.co.uk>
%
% This work may be distributed and/or modified under the
% conditions of the LaTeX Project Public License, either
% version 1.3c of this license or (at your option) any later
% version. The latest version of this license is in
%    http://www.latex-project.org/lppl.txt
% and version 1.3c or later is part of all distributions of
% LaTeX version 2005/12/01 or later.
%
% This work has the LPPL maintenance status `maintained'.
%
% The current maintainer of this work is Joseph Wright.
%
% This work consists of the source file pgfopts.dtx
%                 and the derived files pgfopts.ins,
%                                       pgfopts.sty,
%                                       pgfopts.pdf,
%                                       pgfopts-manual.pdf,
%                                       jawltxdoc.sty and
%                                       README
%
% TDS-ready files:
%    The compressed file pgfopts.tds.zip contains an unpacked version
%    of all of the files included here, and pre-compiled
%    documentation in PDF format.  Simply decompress pgfopts.tds.dtx
%    in your local TeX directory, run your hash program (texhash,
%    initexmf --update-fndb, etc.) and everything will be ready to
%    go.  The user documentation for the package is called
%    pgfopts-manual.pdf; the file pgfopts.pdf includes the user
%    manual and the fully-indexed source code.
%
% Unpacking:
%    (a) If pgfopts.ins is present:
%           tex pgfopts.ins
%    (b) Without pgfopts.ins:
%           tex pgfopts.dtx
%    (c) If you use LaTeX to generate files:
%           latex \let\install=y% \iffalse meta-comment
%
% Copyright (C) 2008 by
%    Joseph Wright <joseph.wright@morningstar2.co.uk>
%
% This work may be distributed and/or modified under the
% conditions of the LaTeX Project Public License, either
% version 1.3c of this license or (at your option) any later
% version. The latest version of this license is in
%    http://www.latex-project.org/lppl.txt
% and version 1.3c or later is part of all distributions of
% LaTeX version 2005/12/01 or later.
%
% This work has the LPPL maintenance status `maintained'.
%
% The current maintainer of this work is Joseph Wright.
%
% This work consists of the source file pgfopts.dtx
%                 and the derived files pgfopts.ins,
%                                       pgfopts.sty,
%                                       pgfopts.pdf,
%                                       pgfopts-manual.pdf,
%                                       jawltxdoc.sty and
%                                       README
%
% TDS-ready files:
%    The compressed file pgfopts.tds.zip contains an unpacked version
%    of all of the files included here, and pre-compiled
%    documentation in PDF format.  Simply decompress pgfopts.tds.dtx
%    in your local TeX directory, run your hash program (texhash,
%    initexmf --update-fndb, etc.) and everything will be ready to
%    go.  The user documentation for the package is called
%    pgfopts-manual.pdf; the file pgfopts.pdf includes the user
%    manual and the fully-indexed source code.
%
% Unpacking:
%    (a) If pgfopts.ins is present:
%           tex pgfopts.ins
%    (b) Without pgfopts.ins:
%           tex pgfopts.dtx
%    (c) If you use LaTeX to generate files:
%           latex \let\install=y% \iffalse meta-comment
%
% Copyright (C) 2008 by
%    Joseph Wright <joseph.wright@morningstar2.co.uk>
%
% This work may be distributed and/or modified under the
% conditions of the LaTeX Project Public License, either
% version 1.3c of this license or (at your option) any later
% version. The latest version of this license is in
%    http://www.latex-project.org/lppl.txt
% and version 1.3c or later is part of all distributions of
% LaTeX version 2005/12/01 or later.
%
% This work has the LPPL maintenance status `maintained'.
%
% The current maintainer of this work is Joseph Wright.
%
% This work consists of the source file pgfopts.dtx
%                 and the derived files pgfopts.ins,
%                                       pgfopts.sty,
%                                       pgfopts.pdf,
%                                       pgfopts-manual.pdf,
%                                       jawltxdoc.sty and
%                                       README
%
% TDS-ready files:
%    The compressed file pgfopts.tds.zip contains an unpacked version
%    of all of the files included here, and pre-compiled
%    documentation in PDF format.  Simply decompress pgfopts.tds.dtx
%    in your local TeX directory, run your hash program (texhash,
%    initexmf --update-fndb, etc.) and everything will be ready to
%    go.  The user documentation for the package is called
%    pgfopts-manual.pdf; the file pgfopts.pdf includes the user
%    manual and the fully-indexed source code.
%
% Unpacking:
%    (a) If pgfopts.ins is present:
%           tex pgfopts.ins
%    (b) Without pgfopts.ins:
%           tex pgfopts.dtx
%    (c) If you use LaTeX to generate files:
%           latex \let\install=y% \iffalse meta-comment
%
% Copyright (C) 2008 by
%    Joseph Wright <joseph.wright@morningstar2.co.uk>
%
% This work may be distributed and/or modified under the
% conditions of the LaTeX Project Public License, either
% version 1.3c of this license or (at your option) any later
% version. The latest version of this license is in
%    http://www.latex-project.org/lppl.txt
% and version 1.3c or later is part of all distributions of
% LaTeX version 2005/12/01 or later.
%
% This work has the LPPL maintenance status `maintained'.
%
% The current maintainer of this work is Joseph Wright.
%
% This work consists of the source file pgfopts.dtx
%                 and the derived files pgfopts.ins,
%                                       pgfopts.sty,
%                                       pgfopts.pdf,
%                                       pgfopts-manual.pdf,
%                                       jawltxdoc.sty and
%                                       README
%
% TDS-ready files:
%    The compressed file pgfopts.tds.zip contains an unpacked version
%    of all of the files included here, and pre-compiled
%    documentation in PDF format.  Simply decompress pgfopts.tds.dtx
%    in your local TeX directory, run your hash program (texhash,
%    initexmf --update-fndb, etc.) and everything will be ready to
%    go.  The user documentation for the package is called
%    pgfopts-manual.pdf; the file pgfopts.pdf includes the user
%    manual and the fully-indexed source code.
%
% Unpacking:
%    (a) If pgfopts.ins is present:
%           tex pgfopts.ins
%    (b) Without pgfopts.ins:
%           tex pgfopts.dtx
%    (c) If you use LaTeX to generate files:
%           latex \let\install=y\input{pgfopts.dtx}
%
% Documentation:
%    (a) Without write18 enabled:
%          pdflatex pgfopts.dtx
%          makeindex -s gind.ist pgfopts.idx
%          makeindex -s gglo.ist -o pgfopts.gls  pgfopts.glo
%          pdflatex pgfopts.dtx
%          pdflatex pgfopts.dtx
%    (b) With write18 enabled:
%          pdflatex pgfopts.dtx
%          pdflatex pgfopts.dtx
%          pdflatex pgfopts.dtx
%
% Installation:
%     Copy pgfopts.sty to a location searched by TeX, and if required
%     by your TeX installation, run the appropriate command to build
%     a hash of files (texhash, initexmf --update-fndb, etc.)
%
% Note:
%     The jawltxdoc.sty file is not needed for installation,
%     only for building the documentation; it may be deleted
%     after producing the documentation (if necessary).
%
%<*ignore>
% This is all taken verbatim from Heiko Oberdiek's packages
\begingroup
  \def\x{LaTeX2e}%
\expandafter\endgroup
\ifcase 0\ifx\install y1\fi\expandafter
         \ifx\csname processbatchFile\endcsname\relax\else1\fi
         \ifx\fmtname\x\else 1\fi\relax
\else\csname fi\endcsname
%</ignore>
%<*install>
\input docstrip.tex
\keepsilent
\askforoverwritefalse
\preamble
 --------------------------------------------------------------
 The pgfopts package --- LaTeX package options with pgfkeys
 Maintained by Joseph Wright
 E-mail: joseph.wright@morningstar2.co.uk
 Released under the LaTeX Project Public License v1.3 or later
 See http://www.latex-project.org/lppl.txt
 --------------------------------------------------------------

\endpreamble
\Msg{Generating pgfopts files:}
\generate{\file{jawltxdoc.sty}{\from{\jobname.dtx}{jawltxdoc}}
}
\usedir{tex/latex/pgfopts}
\generate{\file{\jobname.sty}{\from{\jobname.dtx}{package}}
}
\usedir{source/latex/pgfopts}
\generate{\file{\jobname.ins}{\from{\jobname.dtx}{install}}
}
\nopreamble\nopostamble
\usedir{doc/latex/pgfopts}
\generate{\file{README.txt}{\from{\jobname.dtx}{readme}}
}
\endbatchfile
%</install>
%<*readme>
---------------------------------------------------------------
The pgfopts package --- LaTeX package options with pgfkeys
Maintained by Joseph Wright
E-mail: joseph.wright@morningstar2.co.uk
Released under the LaTeX Project Public License v1.3c or later
See http://www.latex-project.org/lppl.txt
---------------------------------------------------------------

Using key-value options for packages and macros is a good way
of handling large numbers of options with a clean interface.
The pgfkeys package provides a very well designed system for
defining and using keys, but does not make this available for
handling LaTeX class and package options.  The pgfopts package
adds this ability to pgfkeys, in the same way that kvoptions
extends the keyval package.

%</readme>
%<*ignore>
\fi
% Will Robertson's trick
\immediate\write18{makeindex -s gind.ist -o \jobname.ind  \jobname.idx}
\immediate\write18{makeindex -s gglo.ist -o \jobname.gls  \jobname.glo}
%</ignore>
%<*driver>
\PassOptionsToClass{a4paper}{article}
\documentclass[german,english,UKenglish]{ltxdoc}
\EnableCrossrefs
\CodelineIndex
\RecordChanges
%\OnlyDescription
\usepackage{jawltxdoc}
\begin{document}
  \DocInput{\jobname.dtx}
\end{document}
%</driver>
% \fi
%
%\CheckSum{220}
%
% \CharacterTable
%  {Upper-case    \A\B\C\D\E\F\G\H\I\J\K\L\M\N\O\P\Q\R\S\T\U\V\W\X\Y\Z
%   Lower-case    \a\b\c\d\e\f\g\h\i\j\k\l\m\n\o\p\q\r\s\t\u\v\w\x\y\z
%   Digits        \0\1\2\3\4\5\6\7\8\9
%   Exclamation   \!     Double quote  \"     Hash (number) \#
%   Dollar        \$     Percent       \%     Ampersand     \&
%   Acute accent  \'     Left paren    \(     Right paren   \)
%   Asterisk      \*     Plus          \+     Comma         \,
%   Minus         \-     Point         \.     Solidus       \/
%   Colon         \:     Semicolon     \;     Less than     \<
%   Equals        \=     Greater than  \>     Question mark \?
%   Commercial at \@     Left bracket  \[     Backslash     \\
%   Right bracket \]     Circumflex    \^     Underscore    \_
%   Grave accent  \`     Left brace    \{     Vertical bar  \|
%   Right brace   \}     Tilde         \~}
%
% \GetFileInfo{\jobname.sty}
%
%\changes{v1.0}{2008/06/26}{First public release}
%
%\setkeys{lst}{language=[LaTeX]{TeX},moretexcs={pgfkeys,
%  ProcessPgfOptions,pgfkeyscurrentname,pgfkeyscurrentvalue}}
%
%\DoNotIndex{\.,\/,\:,\=,\@classoptionslist,\@clsextension}
%\DoNotIndex{\@currext,\@currname,\@empty,\@expandtwoargs,\@for}
%\DoNotIndex{\@ifstar,\@ifundefined,\@makeother,\@nil,\@onlypreamble}
%\DoNotIndex{\@removeelement,\@tempa,\@unprocessedoptions}
%\DoNotIndex{\@unusedoptionlist,\AtEndOfPackage,\begingroup}
%\DoNotIndex{\catcode,\csname,\CurrentOption,\def,\do,\edef}
%\DoNotIndex{\else,\endcsname,\endgroup,\expandafter,\fi}
%\DoNotIndex{\gdef,\global,\ifx,\let,\newcommand,\noexpand}
%\DoNotIndex{\NeedsTeXFormat,\pgfkeys,\pgfkeysifdefined}
%\DoNotIndex{\ProvidesPackage,\relax,\RequirePackage,\space}
%\DoNotIndex{\string,\the,\toks@}
%
% \title{\currpkg\ --- \LaTeX\ package options with pgfkeys^^A
%   \thanks{This file describes version \fileversion, last revised
%           \filedate.}}
% \author{Joseph Wright^^A
%   \thanks{E-mail: joseph.wright@morningstar2.co.uk}}
% \date{Released \filedate}
%
% \maketitle
%
%\begin{abstract}
% Using key--value options for packages and macros is a good way of
% handling large numbers of options with a clean interface. The
% \pkg{pgfkeys} package provides a very well designed system for
% defining and using keys, but does not make this available for
% handling \LaTeX\ class and package options.  The \currpkg package
% adds this ability to \pkg{pgfkeys}, in the same way that
% \pkg{kvoptions} extends the \pkg{keyval} package.
%\end{abstract}
%
%\begin{multicols}{2}
%\tableofcontents
%\end{multicols}
%
%\section{Introduction}
% The key--value method for optional arguments is very popular, as it
% allows the class or package author to define a large number of
% options with a simple interface.  A number of packages can be used
% to provide the key management system, most of which load or extent
% the parent \pkg{keyval} package.  On its own, \pkg{keyval} can only
% be used for parsing the arguments of macros.  However, a number of
% packages have extended the method to processing \LaTeX\ class and
% package options.  This processing is made available as part of two
% general-purpose packages \pkg{xkeyval} and \pkg{kvoptions}; both
% allow the author of a class or package to process key--value
% options given at load-time.
%
% The \pkg{pgfkeys} package provides a somewhat different key--value
% system to \pkg{keyval} and derivatives.  This uses a completely
% different model for defining and using keys, although for the
% end-user the result appears very similar.  The \currpkg package
% allows keys defined with \pkg{pgfkeys} to be used as class or
% package options, in the same way that \pkg{kvoptions} extends
% \pkg{keyval}.
%
% Users of \currpkg should be familiar with the general methods used
% by \pkg{pgfkeys}.  These are outlined in the manual for the
% Ti\emph{k}z and \pkg{pgf} bundle.
%
%\section{Using the package}
%\subsection{Creating options}
% To create package or class options for use with \currpkg, it is
% only necessary to define the appropriate keys.  Taking as an
% example a package ``\pkg{MyOwnPackage}'', which uses the prefix
% \texttt{MOP} on internal macros, creating keys would take the form:
%\begin{LaTeXexample}[noexample]
%  \pgfkeys{/MOP/.cd,
%    keyone/.code=\wlog{Value `#1' given},
%    keytwo/.store in=\MOP@store}
%\end{LaTeXexample}
% Here, \opt{keyone} simply writes its argument to the log, while
% \opt{keytwo} stores the value given in the \cs{MOP@store} macro.
%
% An important point to notice is that the key names \emph{do not}
% contain a space. This is because the \LaTeX\ kernel removes spaces
% from options before they are passed to the class or package. Spaces
% can occur in the path to the key, but not in the key name itself.
% This restriction only applies to keys used as options.
%
%\subsection{Processing options}
%\DescribeMacro{\ProcessPgfOptions}
% Options should be processed using the \cs{ProcessPgfOptions} macro,
% which takes two forms.  When invoked followed by a star, the system
% will use the name of the current file as the key path.  Thus, for
% the example \pkg{MyOwnPackage.sty}, \currpkg would use the path
% \opt{/MyOwnPackage}.
%\begin{LaTeXexample}[noexample]
%  \ProcessPgfOptions*
%\end{LaTeXexample}
% When the key path does not match the current file name, the second
% form of \cs{ProcessPgfOptions} should be used.  The full path to be
% searched should be given as the argument to the macro.  Continuing
% the example, for \pkg{MyOwnPackage.sty} using path \opt{/MOP}, the
% correct call is:
%\begin{LaTeXexample}[noexample]
%  \ProcessPgfOptions{/MOP}
%\end{LaTeXexample}
% Notice that the \emph{full} path is given here, including the
% leading ``|/|''; this allows the advance user to alter the default
% path before calling \cs{ProcessPgfOptions}.
%
%\subsection{Specialised options}
% Users of \pkg{kvoptions} will be aware that there are two types of
% options which need specialised handling.  First, there is the need
% to process options which should not be given a value. Using
% \pkg{pgfkeys}, this is achieved using the \opt{.value forbidden}
% handler.
%\begin{LaTeXexample}[noexample]
%  \pgfkeys{/MOP/.cd,
%    keythree/.code=\wlog{keythree set},
%    keythree/.value forbidden}
%\end{LaTeXexample}
%
% The second type of specialist option is the unknown option.
% Some packages accept arbitrary values in the options, which
% can then be used to load configuration files, \etc.  Once
% again, this is handled using correctly-defined keys.
%\begin{LaTeXexample}[noexample]
%  \pgfkeys{/MOP/.cd,
%    .unknown/.code=\wlog{%
%      The unknown key `\pgfkeyscurrentname' was given with
%      value `\pgfkeyscurrentvalue'}}
%\end{LaTeXexample}
%
% The power of the \pkg{pgfkeys} system is demonstrated here: no
% additional macros are needed to handle the specialised option
% types.
%
% \StopEventually{^^A
%   \PrintChanges
%   \PrintIndex}
%
% \iffalse
%<*package>
% \fi
%
% \section{Implementation}
% The code here is based heavily on \pkg{kvoptions}, which has a
% similar aim to \currpkg, but works with \pkg{keyval} and derived
% packages.
%
%\begin{macro}{\pgfopts@id}
% Macro naming follows the scheme used by \pkg{pgfkeys}: the full
% package name is the prefix.
%    \begin{macrocode}
\NeedsTeXFormat{LaTeX2e}
\newcommand*{\pgfopts@id}{}
\def\pgfopts@id$#1: #2.#3 #4 #5-#6-#7 #8 #9${%
    #5/#6/#7\space v1.0\space}
\ProvidesPackage{pgfopts}
  [\pgfopts@id $Id: pgfopts.dtx 30 2008-06-26 19:27:20Z joseph $
   LaTeX package options with pgfkeys]
%    \end{macrocode}
%\end{macro}
% The \pkg{pgfkeys} package is needed, unsurprisingly.
%    \begin{macrocode}
\RequirePackage{pgfkeys}
%    \end{macrocode}
%
%\begin{macro}{\pgfopts@catcodes}
% A few category codes are set correctly, to prevent odd
% accidents.
%    \begin{macrocode}
\edef\pgfopts@catcodes{%
  \catcode\string`\string ` \the\catcode\string`\`\relax
  \catcode\string`\string = \the\catcode\string`\=\relax
  \catcode\string`\string : \the\catcode\string`\:\relax
  \catcode\string`\string , \the\catcode\string`\,\relax
  \catcode\string`\string / \the\catcode\string`\/\relax
  \catcode\string`\string . \the\catcode\string`\.\relax}
\catcode\string`\` 12\relax
\catcode`\= 12\relax
\@makeother{\:}
\@makeother{\,}
\@makeother{\/}
\@makeother{\.}
%    \end{macrocode}
%\end{macro}
%
%\begin{macro}{\ProcessPgfOptions}
% The user macro for processing options may have a star.  In that
% case, \cs{@currname} is used for setting keys.
%    \begin{macrocode}
\newcommand*{\ProcessPgfOptions}{%
  \@ifstar
    {\begingroup
       \edef\@tempa{%
         \endgroup
         \noexpand\pgfopts@ProcessOptions
           {/\@currname}}%
     \@tempa}%
    {\pgfopts@ProcessOptions}}
\@onlypreamble\ProcessPgfOptions
%    \end{macrocode}
%\end{macro}
%\begin{macro}{\pgfopts@temp}
% A private storage area is needed by the next macro.
%    \begin{macrocode}
\newcommand*{\pgfopts@temp}{}
%    \end{macrocode}
%\end{macro}
%\begin{macro}{\pgfopts@ProcessOptions}
%  \darg{path}
% The business end of the package.  This is a modified copy of the
% macro \cs{KVO@ProcessOptions} from \pkg{kvoptions}.  Some
% simplifications are made due to the differences in key system, and
% as patching the kernel is ignored.
%    \begin{macrocode}
\newcommand*{\pgfopts@ProcessOptions}[1]{%
  \def\pgfopts@temp{,}%
  \ifx\@currext\@clsextension\else
    \ifx\@classoptionslist\relax\else
%    \end{macrocode}
%\begin{macro}{\pgfopts@CurrentOption}
% For packages, the list of global (class) options needs to be
% checked. The assumption is made that only \emph{explicitly}
% declared keys will be accepted as options here. The strange
% argument to \cs{pgfkeysifdefined} is needed as checking for the key
% itself only works if it has already been used.
%    \begin{macrocode}
      \@for\pgfopts@CurrentOption:=\@classoptionslist\do{%
        \pgfkeysifdefined{#1/\expandafter\pgfopts@sepkeyval%
          \pgfopts@CurrentOption=\@nil/.@cmd}
          {\edef\pgfopts@temp{%
             \pgfopts@temp\pgfopts@CurrentOption,}%
           \@expandtwoargs\@removeelement\pgfopts@CurrentOption
             \@unusedoptionlist\@unusedoptionlist}%
          {}}%
    \fi
  \fi
%    \end{macrocode}
%\end{macro}
% Now the local options have to be processed.  If there are none,
% then the global list is simply copied.  Otherwise, the local
% options are first added to the ``to do'' list stored in \cs{toks@}.
%    \begin{macrocode}
  \begingroup
    \@ifundefined{opt@\@currname.\@currext}
      {\toks@\expandafter{\pgfopts@temp}}
      {\toks@\expandafter\expandafter\expandafter{%
         \csname opt@\@currname.\@currext\endcsname}%
%    \end{macrocode}
% There is work to do if processing options of a class; unknown options
% from the global list have to be completely ignored.
%    \begin{macrocode}
       \ifx\@currext\@clsextension
         \edef\@tempa{\the\toks@}%
         \toks@\expandafter{\pgfopts@temp}%
         \@for\CurrentOption:=\@tempa\do{%
           \pgfkeysifdefined{#1/\expandafter%
             \pgfopts@sepkeyval\CurrentOption=\@nil/.@cmd}
             {\toks@\expandafter{\the\expandafter\toks@
                \expandafter,\CurrentOption}}
             {\ifx\@empty\@unusedoptionlist\@empty
                \global\let\@unusedoptionlist\CurrentOption
              \else
                \expandafter\expandafter\expandafter\gdef
                \expandafter\expandafter\expandafter
                \@unusedoptionlist\expandafter\expandafter
                \expandafter{%
                  \expandafter\@unusedoptionlist
                  \expandafter,\CurrentOption}%
              \fi}}%
%    \end{macrocode}
% For packages, the two lists are simply combined.
%    \begin{macrocode}
      \else
        \toks@\expandafter\expandafter\expandafter{%
            \expandafter\pgfopts@temp\the\toks@}%
      \fi}%
%    \end{macrocode}
% With everything done, the keys are sent to \pkg{pgfkeys} for
% processing.  The default key system in \pkg{pgfkeys} is very
% different from that in \pkg{kvoptions}, and so the system here is
% much less complex.
%    \begin{macrocode}
    \edef\pgfopts@temp{\endgroup
      \noexpand\pgfkeys{#1/.cd,\the\toks@}}%
  \pgfopts@temp
  \let\CurrentOption\@empty
  \AtEndOfPackage{\let\@unprocessedoptions\relax}}
%    \end{macrocode}
%\end{macro}
%\begin{macro}{\pgfopts@sepkeyval}
% A very simply macro to separate a key from a potential value.
%    \begin{macrocode}
\newcommand*{\pgfopts@sepkeyval}{}
\def\pgfopts@sepkeyval#1=#2\@nil{#1}
%    \end{macrocode}
%\end{macro}
% Tidying up.
%    \begin{macrocode}
\pgfopts@catcodes
%    \end{macrocode}
%
%\Finale
%
%\iffalse
%</package>
%<*jawltxdoc>
\NeedsTeXFormat{LaTeX2e}
\ProvidesPackage{jawltxdoc}
\usepackage[T1]{fontenc}
\usepackage{lmodern}
\usepackage[final]{listings,graphicx,microtype}
\usepackage[scaled=0.95]{helvet}
\usepackage[version=3]{mhchem}
\usepackage[osf]{mathpazo}
\usepackage{booktabs,array,url,courier,xspace,etoolbox,varioref}
\usepackage{upgreek,ifpdf,float,caption,longtable,babel}
\floatstyle{plaintop}
\restylefloat{table}
\labelformat{figure}{\figurename~#1}
\labelformat{table}{\tablename~#1}
\ifpdf
  \usepackage{embedfile}
  \embedfile[%
    stringmethod=escape,%
    mimetype=plain/text,%
    desc={LaTeX docstrip source archive for package `\jobname'}%
    ]{\jobname.dtx}
\fi
\IfFileExists{\jobname.sty}
  {\usepackage{\jobname}}{}
\usepackage[numbered]{hypdoc}
\setcounter{IndexColumns}{2}
\newlength\LaTeXwidth
\newlength\LaTeXoutdent
\newlength\LaTeXgap
\setlength\LaTeXgap{1em}
\setlength\LaTeXoutdent{-0.15\textwidth}
\newbox\lst@samplebox
\edef\LaTeXexamplefile{\jobname.tmp}
\lst@RequireAspects{writefile}
\lstnewenvironment{LaTeXexample}[1][example]{%
  \global\let\lst@intname\@empty
  \ifcsname LaTeXcode#1\endcsname
    \expandafter\let\expandafter\LaTeXcode
      \csname LaTeXcode#1\endcsname
    \expandafter\let\expandafter\LaTeXcodeend
      \csname LaTeXcode#1end\endcsname
  \else
    \PackageError{jawltxdoc}
      {Undefined example type `#1'}
      \@ehd
    \let\LaTeXcode\relax
    \let\LaTeXcodeend\relax
  \fi
  \LaTeXcode}
  {\lst@EndWriteFile
   \LaTeXcodeend}
\newcommand*{\LaTeXcodeexample}{%
  \setbox\lst@samplebox=\hbox\bgroup
  \LaTeXcodefloat}
\let\LaTeXcoderesultonly\LaTeXcodeexample
\newcommand*{\LaTeXcodeexampleend}{%
  \egroup
  \setlength\LaTeXwidth{\wd\lst@samplebox}%
  \begin{list}{}{%
    \setlength\itemindent{0pt}
    \setlength\leftmargin\LaTeXoutdent
    \setlength\rightmargin{0pt}}%
    \item
      \setlength\LaTeXoutdent{-0.15\textwidth}
      \begin{minipage}[c]{%
        \textwidth-\LaTeXwidth-\LaTeXoutdent-\LaTeXgap}
        \LaTeXcodefloatend
      \end{minipage}%
      \hfill
      \begin{minipage}[c]{\LaTeXwidth}%
        \hbox to\linewidth{\box\lst@samplebox\hss}%
      \end{minipage}%
  \end{list}}
\let\LaTeXcodenoexample\LaTeXcodeexample
\newcommand*{\LaTeXcodenoexampleend}{%
  \egroup
  \setlength\LaTeXwidth{\wd\lst@samplebox}%
  \begin{list}{}{%
    \setlength\itemindent{0pt}
    \setlength\leftmargin{0pt}
    \setlength\rightmargin{0pt}}%
    \item
      \begin{minipage}[c]{\LaTeXwidth}
        \hbox to\linewidth{\box\lst@samplebox\hss}%
      \end{minipage}%
  \end{list}}
\newcommand*{\LaTeXcodefloat}{%
  \setkeys{lst}{tabsize=4,gobble=3,breakindent=0pt,
    basicstyle=\small\ttfamily,basewidth=0.51em,
    keywordstyle=\color{blue}}%
  \lst@BeginAlsoWriteFile{\LaTeXexamplefile}}
\newcommand*{\LaTeXcodefloatend}{%
  \MakePercentComment\catcode`\^^M=10\relax
  \small
  {\setkeys{lst}{SelectCharTable=\lst@ReplaceInput{\^\^I}%
    {\lst@ProcessTabulator}}%
    \leavevmode \input{\LaTeXexamplefile}}%
  \MakePercentIgnore}
\newcommand*{\LaTeXcoderesultonlyend}{\egroup\LaTeXcodefloatend}
\lstnewenvironment{BibTeXexample}{%
  \global\let\lst@intname\@empty
  \setbox\lst@samplebox=\hbox\bgroup
  \setkeys{lst}{tabsize=4,gobble=3,breakindent=0pt,
    basicstyle=\small\ttfamily,basewidth=0.51em,
    keywordstyle=\color{black}}
  \lst@BeginAlsoWriteFile{\LaTeXexamplefile}}
 {\lst@EndWriteFile
   \LaTeXcodeexampleend}
\newcommand*{\DescribeOption}{%
  \leavevmode\@bsphack\begingroup\MakePrivateLetters
  \Describe@Option}
\newcommand*{\Describe@Option}[1]{\endgroup
              \marginpar{\raggedleft\PrintDescribeEnv{#1}}%
              \SpecialOptionIndex{#1}\@esphack\ignorespaces}
\newcommand*{\SpecialOptionIndex}[1]{\@bsphack
    \index{#1\actualchar{\protect\ttfamily#1}
           (option)\encapchar usage}%
    \index{options:\levelchar#1\actualchar{\protect\ttfamily#1}%
      \encapchar usage}\@esphack}
\newcommand*{\indexopt}[1]{\DescribeOption{#1}\opt{#1}}
\newcommand*{\DescribeOptionInfo}[2]{%
  \DescribeOption{#1}%
  \opt{#1=\meta{#2}}\xspace}
\newcommand*{\ofixarg}[1]{%
  {\ttfamily[}%
  \ifmmode \expandafter \nfss@text \fi
  {%
    \meta@font@select
    \edef\meta@hyphen@restore{%
      \hyphenchar\the\font\the\hyphenchar\font}%
    \hyphenchar\font\m@ne
    \language\l@nohyphenation
    #1\/%
    \meta@hyphen@restore
    }%
    {\ttfamily]}}
\newcommand*{\pkg}[1]{\textsf{#1}}
\newcommand*{\currpkg}{\pkg{\jobname}\xspace}
\newcommand*{\opt}[1]{\texttt{#1}}
\newcommand*{\defaultopt}[1]{\opt{\textbf{#1}}}
\newcommand*{\file}[1]{\texttt{#1}}
\newcommand*{\ext}[1]{\file{.#1}}
\newcommand*{\latin}[1]{\emph{#1}}
\newcommand*{\etc}{%
  \@ifnextchar.
    {\latin{etc}}
    {\latin{etc}.\xspace}}
\newcommand*{\eg}{%
  \@ifnextchar.
    {\latin{e.g}}
    {\latin{e.g}.\xspace}}
\newcommand*{\ie}{%
  \@ifnextchar.
    {\latin{i.e}}
    {\latin{i.e}.\xspace}}
\newcommand*{\etal}{%
  \@ifnextchar.
    {\latin{et~al.}}
    {\latin{et~al}.\xspace}}
\newcommand*{\AMS}{{\protect\usefont{OMS}{cmsy}{m}{n}%
  A\kern-.1667em\lower.5ex\hbox{M}\kern-.125emS}}
\providecommand*{\eTeX}{\ensuremath{\varepsilon}-\TeX}
\DeclareRobustCommand*{\XeTeX}
  {X\kern-.125em\lower.5ex\hbox{\reflectbox{E}}\kern-.1667em\TeX}
\providecommand*{\CTAN}{\textsc{ctan}}
\patchcmd{\@addmarginpar}
  {\@latex@warning@no@line {Marginpar on page \thepage\space moved}}
  {\relax}{}{}
\newcounter{argument}
\g@addto@macro\endmacro{\setcounter{argument}{0}}
\newcommand*\darg[1]{%
  \stepcounter{argument}%
  {\ttfamily\char`\#\theargument~:~}#1\par\noindent\ignorespaces}
\newcommand*\doarg[1]{%
  \stepcounter{argument}%
  {\ttfamily\makebox[0pt][r]{[}%
   \char`\#\theargument]:~}#1\par\noindent\ignorespaces}
%</jawltxdoc>
%\fi

%
% Documentation:
%    (a) Without write18 enabled:
%          pdflatex pgfopts.dtx
%          makeindex -s gind.ist pgfopts.idx
%          makeindex -s gglo.ist -o pgfopts.gls  pgfopts.glo
%          pdflatex pgfopts.dtx
%          pdflatex pgfopts.dtx
%    (b) With write18 enabled:
%          pdflatex pgfopts.dtx
%          pdflatex pgfopts.dtx
%          pdflatex pgfopts.dtx
%
% Installation:
%     Copy pgfopts.sty to a location searched by TeX, and if required
%     by your TeX installation, run the appropriate command to build
%     a hash of files (texhash, initexmf --update-fndb, etc.)
%
% Note:
%     The jawltxdoc.sty file is not needed for installation,
%     only for building the documentation; it may be deleted
%     after producing the documentation (if necessary).
%
%<*ignore>
% This is all taken verbatim from Heiko Oberdiek's packages
\begingroup
  \def\x{LaTeX2e}%
\expandafter\endgroup
\ifcase 0\ifx\install y1\fi\expandafter
         \ifx\csname processbatchFile\endcsname\relax\else1\fi
         \ifx\fmtname\x\else 1\fi\relax
\else\csname fi\endcsname
%</ignore>
%<*install>
\input docstrip.tex
\keepsilent
\askforoverwritefalse
\preamble
 --------------------------------------------------------------
 The pgfopts package --- LaTeX package options with pgfkeys
 Maintained by Joseph Wright
 E-mail: joseph.wright@morningstar2.co.uk
 Released under the LaTeX Project Public License v1.3 or later
 See http://www.latex-project.org/lppl.txt
 --------------------------------------------------------------

\endpreamble
\Msg{Generating pgfopts files:}
\generate{\file{jawltxdoc.sty}{\from{\jobname.dtx}{jawltxdoc}}
}
\usedir{tex/latex/pgfopts}
\generate{\file{\jobname.sty}{\from{\jobname.dtx}{package}}
}
\usedir{source/latex/pgfopts}
\generate{\file{\jobname.ins}{\from{\jobname.dtx}{install}}
}
\nopreamble\nopostamble
\usedir{doc/latex/pgfopts}
\generate{\file{README.txt}{\from{\jobname.dtx}{readme}}
}
\endbatchfile
%</install>
%<*readme>
---------------------------------------------------------------
The pgfopts package --- LaTeX package options with pgfkeys
Maintained by Joseph Wright
E-mail: joseph.wright@morningstar2.co.uk
Released under the LaTeX Project Public License v1.3c or later
See http://www.latex-project.org/lppl.txt
---------------------------------------------------------------

Using key-value options for packages and macros is a good way
of handling large numbers of options with a clean interface.
The pgfkeys package provides a very well designed system for
defining and using keys, but does not make this available for
handling LaTeX class and package options.  The pgfopts package
adds this ability to pgfkeys, in the same way that kvoptions
extends the keyval package.

%</readme>
%<*ignore>
\fi
% Will Robertson's trick
\immediate\write18{makeindex -s gind.ist -o \jobname.ind  \jobname.idx}
\immediate\write18{makeindex -s gglo.ist -o \jobname.gls  \jobname.glo}
%</ignore>
%<*driver>
\PassOptionsToClass{a4paper}{article}
\documentclass[german,english,UKenglish]{ltxdoc}
\EnableCrossrefs
\CodelineIndex
\RecordChanges
%\OnlyDescription
\usepackage{jawltxdoc}
\begin{document}
  \DocInput{\jobname.dtx}
\end{document}
%</driver>
% \fi
%
%\CheckSum{220}
%
% \CharacterTable
%  {Upper-case    \A\B\C\D\E\F\G\H\I\J\K\L\M\N\O\P\Q\R\S\T\U\V\W\X\Y\Z
%   Lower-case    \a\b\c\d\e\f\g\h\i\j\k\l\m\n\o\p\q\r\s\t\u\v\w\x\y\z
%   Digits        \0\1\2\3\4\5\6\7\8\9
%   Exclamation   \!     Double quote  \"     Hash (number) \#
%   Dollar        \$     Percent       \%     Ampersand     \&
%   Acute accent  \'     Left paren    \(     Right paren   \)
%   Asterisk      \*     Plus          \+     Comma         \,
%   Minus         \-     Point         \.     Solidus       \/
%   Colon         \:     Semicolon     \;     Less than     \<
%   Equals        \=     Greater than  \>     Question mark \?
%   Commercial at \@     Left bracket  \[     Backslash     \\
%   Right bracket \]     Circumflex    \^     Underscore    \_
%   Grave accent  \`     Left brace    \{     Vertical bar  \|
%   Right brace   \}     Tilde         \~}
%
% \GetFileInfo{\jobname.sty}
%
%\changes{v1.0}{2008/06/26}{First public release}
%
%\setkeys{lst}{language=[LaTeX]{TeX},moretexcs={pgfkeys,
%  ProcessPgfOptions,pgfkeyscurrentname,pgfkeyscurrentvalue}}
%
%\DoNotIndex{\.,\/,\:,\=,\@classoptionslist,\@clsextension}
%\DoNotIndex{\@currext,\@currname,\@empty,\@expandtwoargs,\@for}
%\DoNotIndex{\@ifstar,\@ifundefined,\@makeother,\@nil,\@onlypreamble}
%\DoNotIndex{\@removeelement,\@tempa,\@unprocessedoptions}
%\DoNotIndex{\@unusedoptionlist,\AtEndOfPackage,\begingroup}
%\DoNotIndex{\catcode,\csname,\CurrentOption,\def,\do,\edef}
%\DoNotIndex{\else,\endcsname,\endgroup,\expandafter,\fi}
%\DoNotIndex{\gdef,\global,\ifx,\let,\newcommand,\noexpand}
%\DoNotIndex{\NeedsTeXFormat,\pgfkeys,\pgfkeysifdefined}
%\DoNotIndex{\ProvidesPackage,\relax,\RequirePackage,\space}
%\DoNotIndex{\string,\the,\toks@}
%
% \title{\currpkg\ --- \LaTeX\ package options with pgfkeys^^A
%   \thanks{This file describes version \fileversion, last revised
%           \filedate.}}
% \author{Joseph Wright^^A
%   \thanks{E-mail: joseph.wright@morningstar2.co.uk}}
% \date{Released \filedate}
%
% \maketitle
%
%\begin{abstract}
% Using key--value options for packages and macros is a good way of
% handling large numbers of options with a clean interface. The
% \pkg{pgfkeys} package provides a very well designed system for
% defining and using keys, but does not make this available for
% handling \LaTeX\ class and package options.  The \currpkg package
% adds this ability to \pkg{pgfkeys}, in the same way that
% \pkg{kvoptions} extends the \pkg{keyval} package.
%\end{abstract}
%
%\begin{multicols}{2}
%\tableofcontents
%\end{multicols}
%
%\section{Introduction}
% The key--value method for optional arguments is very popular, as it
% allows the class or package author to define a large number of
% options with a simple interface.  A number of packages can be used
% to provide the key management system, most of which load or extent
% the parent \pkg{keyval} package.  On its own, \pkg{keyval} can only
% be used for parsing the arguments of macros.  However, a number of
% packages have extended the method to processing \LaTeX\ class and
% package options.  This processing is made available as part of two
% general-purpose packages \pkg{xkeyval} and \pkg{kvoptions}; both
% allow the author of a class or package to process key--value
% options given at load-time.
%
% The \pkg{pgfkeys} package provides a somewhat different key--value
% system to \pkg{keyval} and derivatives.  This uses a completely
% different model for defining and using keys, although for the
% end-user the result appears very similar.  The \currpkg package
% allows keys defined with \pkg{pgfkeys} to be used as class or
% package options, in the same way that \pkg{kvoptions} extends
% \pkg{keyval}.
%
% Users of \currpkg should be familiar with the general methods used
% by \pkg{pgfkeys}.  These are outlined in the manual for the
% Ti\emph{k}z and \pkg{pgf} bundle.
%
%\section{Using the package}
%\subsection{Creating options}
% To create package or class options for use with \currpkg, it is
% only necessary to define the appropriate keys.  Taking as an
% example a package ``\pkg{MyOwnPackage}'', which uses the prefix
% \texttt{MOP} on internal macros, creating keys would take the form:
%\begin{LaTeXexample}[noexample]
%  \pgfkeys{/MOP/.cd,
%    keyone/.code=\wlog{Value `#1' given},
%    keytwo/.store in=\MOP@store}
%\end{LaTeXexample}
% Here, \opt{keyone} simply writes its argument to the log, while
% \opt{keytwo} stores the value given in the \cs{MOP@store} macro.
%
% An important point to notice is that the key names \emph{do not}
% contain a space. This is because the \LaTeX\ kernel removes spaces
% from options before they are passed to the class or package. Spaces
% can occur in the path to the key, but not in the key name itself.
% This restriction only applies to keys used as options.
%
%\subsection{Processing options}
%\DescribeMacro{\ProcessPgfOptions}
% Options should be processed using the \cs{ProcessPgfOptions} macro,
% which takes two forms.  When invoked followed by a star, the system
% will use the name of the current file as the key path.  Thus, for
% the example \pkg{MyOwnPackage.sty}, \currpkg would use the path
% \opt{/MyOwnPackage}.
%\begin{LaTeXexample}[noexample]
%  \ProcessPgfOptions*
%\end{LaTeXexample}
% When the key path does not match the current file name, the second
% form of \cs{ProcessPgfOptions} should be used.  The full path to be
% searched should be given as the argument to the macro.  Continuing
% the example, for \pkg{MyOwnPackage.sty} using path \opt{/MOP}, the
% correct call is:
%\begin{LaTeXexample}[noexample]
%  \ProcessPgfOptions{/MOP}
%\end{LaTeXexample}
% Notice that the \emph{full} path is given here, including the
% leading ``|/|''; this allows the advance user to alter the default
% path before calling \cs{ProcessPgfOptions}.
%
%\subsection{Specialised options}
% Users of \pkg{kvoptions} will be aware that there are two types of
% options which need specialised handling.  First, there is the need
% to process options which should not be given a value. Using
% \pkg{pgfkeys}, this is achieved using the \opt{.value forbidden}
% handler.
%\begin{LaTeXexample}[noexample]
%  \pgfkeys{/MOP/.cd,
%    keythree/.code=\wlog{keythree set},
%    keythree/.value forbidden}
%\end{LaTeXexample}
%
% The second type of specialist option is the unknown option.
% Some packages accept arbitrary values in the options, which
% can then be used to load configuration files, \etc.  Once
% again, this is handled using correctly-defined keys.
%\begin{LaTeXexample}[noexample]
%  \pgfkeys{/MOP/.cd,
%    .unknown/.code=\wlog{%
%      The unknown key `\pgfkeyscurrentname' was given with
%      value `\pgfkeyscurrentvalue'}}
%\end{LaTeXexample}
%
% The power of the \pkg{pgfkeys} system is demonstrated here: no
% additional macros are needed to handle the specialised option
% types.
%
% \StopEventually{^^A
%   \PrintChanges
%   \PrintIndex}
%
% \iffalse
%<*package>
% \fi
%
% \section{Implementation}
% The code here is based heavily on \pkg{kvoptions}, which has a
% similar aim to \currpkg, but works with \pkg{keyval} and derived
% packages.
%
%\begin{macro}{\pgfopts@id}
% Macro naming follows the scheme used by \pkg{pgfkeys}: the full
% package name is the prefix.
%    \begin{macrocode}
\NeedsTeXFormat{LaTeX2e}
\newcommand*{\pgfopts@id}{}
\def\pgfopts@id$#1: #2.#3 #4 #5-#6-#7 #8 #9${%
    #5/#6/#7\space v1.0\space}
\ProvidesPackage{pgfopts}
  [\pgfopts@id $Id: pgfopts.dtx 30 2008-06-26 19:27:20Z joseph $
   LaTeX package options with pgfkeys]
%    \end{macrocode}
%\end{macro}
% The \pkg{pgfkeys} package is needed, unsurprisingly.
%    \begin{macrocode}
\RequirePackage{pgfkeys}
%    \end{macrocode}
%
%\begin{macro}{\pgfopts@catcodes}
% A few category codes are set correctly, to prevent odd
% accidents.
%    \begin{macrocode}
\edef\pgfopts@catcodes{%
  \catcode\string`\string ` \the\catcode\string`\`\relax
  \catcode\string`\string = \the\catcode\string`\=\relax
  \catcode\string`\string : \the\catcode\string`\:\relax
  \catcode\string`\string , \the\catcode\string`\,\relax
  \catcode\string`\string / \the\catcode\string`\/\relax
  \catcode\string`\string . \the\catcode\string`\.\relax}
\catcode\string`\` 12\relax
\catcode`\= 12\relax
\@makeother{\:}
\@makeother{\,}
\@makeother{\/}
\@makeother{\.}
%    \end{macrocode}
%\end{macro}
%
%\begin{macro}{\ProcessPgfOptions}
% The user macro for processing options may have a star.  In that
% case, \cs{@currname} is used for setting keys.
%    \begin{macrocode}
\newcommand*{\ProcessPgfOptions}{%
  \@ifstar
    {\begingroup
       \edef\@tempa{%
         \endgroup
         \noexpand\pgfopts@ProcessOptions
           {/\@currname}}%
     \@tempa}%
    {\pgfopts@ProcessOptions}}
\@onlypreamble\ProcessPgfOptions
%    \end{macrocode}
%\end{macro}
%\begin{macro}{\pgfopts@temp}
% A private storage area is needed by the next macro.
%    \begin{macrocode}
\newcommand*{\pgfopts@temp}{}
%    \end{macrocode}
%\end{macro}
%\begin{macro}{\pgfopts@ProcessOptions}
%  \darg{path}
% The business end of the package.  This is a modified copy of the
% macro \cs{KVO@ProcessOptions} from \pkg{kvoptions}.  Some
% simplifications are made due to the differences in key system, and
% as patching the kernel is ignored.
%    \begin{macrocode}
\newcommand*{\pgfopts@ProcessOptions}[1]{%
  \def\pgfopts@temp{,}%
  \ifx\@currext\@clsextension\else
    \ifx\@classoptionslist\relax\else
%    \end{macrocode}
%\begin{macro}{\pgfopts@CurrentOption}
% For packages, the list of global (class) options needs to be
% checked. The assumption is made that only \emph{explicitly}
% declared keys will be accepted as options here. The strange
% argument to \cs{pgfkeysifdefined} is needed as checking for the key
% itself only works if it has already been used.
%    \begin{macrocode}
      \@for\pgfopts@CurrentOption:=\@classoptionslist\do{%
        \pgfkeysifdefined{#1/\expandafter\pgfopts@sepkeyval%
          \pgfopts@CurrentOption=\@nil/.@cmd}
          {\edef\pgfopts@temp{%
             \pgfopts@temp\pgfopts@CurrentOption,}%
           \@expandtwoargs\@removeelement\pgfopts@CurrentOption
             \@unusedoptionlist\@unusedoptionlist}%
          {}}%
    \fi
  \fi
%    \end{macrocode}
%\end{macro}
% Now the local options have to be processed.  If there are none,
% then the global list is simply copied.  Otherwise, the local
% options are first added to the ``to do'' list stored in \cs{toks@}.
%    \begin{macrocode}
  \begingroup
    \@ifundefined{opt@\@currname.\@currext}
      {\toks@\expandafter{\pgfopts@temp}}
      {\toks@\expandafter\expandafter\expandafter{%
         \csname opt@\@currname.\@currext\endcsname}%
%    \end{macrocode}
% There is work to do if processing options of a class; unknown options
% from the global list have to be completely ignored.
%    \begin{macrocode}
       \ifx\@currext\@clsextension
         \edef\@tempa{\the\toks@}%
         \toks@\expandafter{\pgfopts@temp}%
         \@for\CurrentOption:=\@tempa\do{%
           \pgfkeysifdefined{#1/\expandafter%
             \pgfopts@sepkeyval\CurrentOption=\@nil/.@cmd}
             {\toks@\expandafter{\the\expandafter\toks@
                \expandafter,\CurrentOption}}
             {\ifx\@empty\@unusedoptionlist\@empty
                \global\let\@unusedoptionlist\CurrentOption
              \else
                \expandafter\expandafter\expandafter\gdef
                \expandafter\expandafter\expandafter
                \@unusedoptionlist\expandafter\expandafter
                \expandafter{%
                  \expandafter\@unusedoptionlist
                  \expandafter,\CurrentOption}%
              \fi}}%
%    \end{macrocode}
% For packages, the two lists are simply combined.
%    \begin{macrocode}
      \else
        \toks@\expandafter\expandafter\expandafter{%
            \expandafter\pgfopts@temp\the\toks@}%
      \fi}%
%    \end{macrocode}
% With everything done, the keys are sent to \pkg{pgfkeys} for
% processing.  The default key system in \pkg{pgfkeys} is very
% different from that in \pkg{kvoptions}, and so the system here is
% much less complex.
%    \begin{macrocode}
    \edef\pgfopts@temp{\endgroup
      \noexpand\pgfkeys{#1/.cd,\the\toks@}}%
  \pgfopts@temp
  \let\CurrentOption\@empty
  \AtEndOfPackage{\let\@unprocessedoptions\relax}}
%    \end{macrocode}
%\end{macro}
%\begin{macro}{\pgfopts@sepkeyval}
% A very simply macro to separate a key from a potential value.
%    \begin{macrocode}
\newcommand*{\pgfopts@sepkeyval}{}
\def\pgfopts@sepkeyval#1=#2\@nil{#1}
%    \end{macrocode}
%\end{macro}
% Tidying up.
%    \begin{macrocode}
\pgfopts@catcodes
%    \end{macrocode}
%
%\Finale
%
%\iffalse
%</package>
%<*jawltxdoc>
\NeedsTeXFormat{LaTeX2e}
\ProvidesPackage{jawltxdoc}
\usepackage[T1]{fontenc}
\usepackage{lmodern}
\usepackage[final]{listings,graphicx,microtype}
\usepackage[scaled=0.95]{helvet}
\usepackage[version=3]{mhchem}
\usepackage[osf]{mathpazo}
\usepackage{booktabs,array,url,courier,xspace,etoolbox,varioref}
\usepackage{upgreek,ifpdf,float,caption,longtable,babel}
\floatstyle{plaintop}
\restylefloat{table}
\labelformat{figure}{\figurename~#1}
\labelformat{table}{\tablename~#1}
\ifpdf
  \usepackage{embedfile}
  \embedfile[%
    stringmethod=escape,%
    mimetype=plain/text,%
    desc={LaTeX docstrip source archive for package `\jobname'}%
    ]{\jobname.dtx}
\fi
\IfFileExists{\jobname.sty}
  {\usepackage{\jobname}}{}
\usepackage[numbered]{hypdoc}
\setcounter{IndexColumns}{2}
\newlength\LaTeXwidth
\newlength\LaTeXoutdent
\newlength\LaTeXgap
\setlength\LaTeXgap{1em}
\setlength\LaTeXoutdent{-0.15\textwidth}
\newbox\lst@samplebox
\edef\LaTeXexamplefile{\jobname.tmp}
\lst@RequireAspects{writefile}
\lstnewenvironment{LaTeXexample}[1][example]{%
  \global\let\lst@intname\@empty
  \ifcsname LaTeXcode#1\endcsname
    \expandafter\let\expandafter\LaTeXcode
      \csname LaTeXcode#1\endcsname
    \expandafter\let\expandafter\LaTeXcodeend
      \csname LaTeXcode#1end\endcsname
  \else
    \PackageError{jawltxdoc}
      {Undefined example type `#1'}
      \@ehd
    \let\LaTeXcode\relax
    \let\LaTeXcodeend\relax
  \fi
  \LaTeXcode}
  {\lst@EndWriteFile
   \LaTeXcodeend}
\newcommand*{\LaTeXcodeexample}{%
  \setbox\lst@samplebox=\hbox\bgroup
  \LaTeXcodefloat}
\let\LaTeXcoderesultonly\LaTeXcodeexample
\newcommand*{\LaTeXcodeexampleend}{%
  \egroup
  \setlength\LaTeXwidth{\wd\lst@samplebox}%
  \begin{list}{}{%
    \setlength\itemindent{0pt}
    \setlength\leftmargin\LaTeXoutdent
    \setlength\rightmargin{0pt}}%
    \item
      \setlength\LaTeXoutdent{-0.15\textwidth}
      \begin{minipage}[c]{%
        \textwidth-\LaTeXwidth-\LaTeXoutdent-\LaTeXgap}
        \LaTeXcodefloatend
      \end{minipage}%
      \hfill
      \begin{minipage}[c]{\LaTeXwidth}%
        \hbox to\linewidth{\box\lst@samplebox\hss}%
      \end{minipage}%
  \end{list}}
\let\LaTeXcodenoexample\LaTeXcodeexample
\newcommand*{\LaTeXcodenoexampleend}{%
  \egroup
  \setlength\LaTeXwidth{\wd\lst@samplebox}%
  \begin{list}{}{%
    \setlength\itemindent{0pt}
    \setlength\leftmargin{0pt}
    \setlength\rightmargin{0pt}}%
    \item
      \begin{minipage}[c]{\LaTeXwidth}
        \hbox to\linewidth{\box\lst@samplebox\hss}%
      \end{minipage}%
  \end{list}}
\newcommand*{\LaTeXcodefloat}{%
  \setkeys{lst}{tabsize=4,gobble=3,breakindent=0pt,
    basicstyle=\small\ttfamily,basewidth=0.51em,
    keywordstyle=\color{blue}}%
  \lst@BeginAlsoWriteFile{\LaTeXexamplefile}}
\newcommand*{\LaTeXcodefloatend}{%
  \MakePercentComment\catcode`\^^M=10\relax
  \small
  {\setkeys{lst}{SelectCharTable=\lst@ReplaceInput{\^\^I}%
    {\lst@ProcessTabulator}}%
    \leavevmode \input{\LaTeXexamplefile}}%
  \MakePercentIgnore}
\newcommand*{\LaTeXcoderesultonlyend}{\egroup\LaTeXcodefloatend}
\lstnewenvironment{BibTeXexample}{%
  \global\let\lst@intname\@empty
  \setbox\lst@samplebox=\hbox\bgroup
  \setkeys{lst}{tabsize=4,gobble=3,breakindent=0pt,
    basicstyle=\small\ttfamily,basewidth=0.51em,
    keywordstyle=\color{black}}
  \lst@BeginAlsoWriteFile{\LaTeXexamplefile}}
 {\lst@EndWriteFile
   \LaTeXcodeexampleend}
\newcommand*{\DescribeOption}{%
  \leavevmode\@bsphack\begingroup\MakePrivateLetters
  \Describe@Option}
\newcommand*{\Describe@Option}[1]{\endgroup
              \marginpar{\raggedleft\PrintDescribeEnv{#1}}%
              \SpecialOptionIndex{#1}\@esphack\ignorespaces}
\newcommand*{\SpecialOptionIndex}[1]{\@bsphack
    \index{#1\actualchar{\protect\ttfamily#1}
           (option)\encapchar usage}%
    \index{options:\levelchar#1\actualchar{\protect\ttfamily#1}%
      \encapchar usage}\@esphack}
\newcommand*{\indexopt}[1]{\DescribeOption{#1}\opt{#1}}
\newcommand*{\DescribeOptionInfo}[2]{%
  \DescribeOption{#1}%
  \opt{#1=\meta{#2}}\xspace}
\newcommand*{\ofixarg}[1]{%
  {\ttfamily[}%
  \ifmmode \expandafter \nfss@text \fi
  {%
    \meta@font@select
    \edef\meta@hyphen@restore{%
      \hyphenchar\the\font\the\hyphenchar\font}%
    \hyphenchar\font\m@ne
    \language\l@nohyphenation
    #1\/%
    \meta@hyphen@restore
    }%
    {\ttfamily]}}
\newcommand*{\pkg}[1]{\textsf{#1}}
\newcommand*{\currpkg}{\pkg{\jobname}\xspace}
\newcommand*{\opt}[1]{\texttt{#1}}
\newcommand*{\defaultopt}[1]{\opt{\textbf{#1}}}
\newcommand*{\file}[1]{\texttt{#1}}
\newcommand*{\ext}[1]{\file{.#1}}
\newcommand*{\latin}[1]{\emph{#1}}
\newcommand*{\etc}{%
  \@ifnextchar.
    {\latin{etc}}
    {\latin{etc}.\xspace}}
\newcommand*{\eg}{%
  \@ifnextchar.
    {\latin{e.g}}
    {\latin{e.g}.\xspace}}
\newcommand*{\ie}{%
  \@ifnextchar.
    {\latin{i.e}}
    {\latin{i.e}.\xspace}}
\newcommand*{\etal}{%
  \@ifnextchar.
    {\latin{et~al.}}
    {\latin{et~al}.\xspace}}
\newcommand*{\AMS}{{\protect\usefont{OMS}{cmsy}{m}{n}%
  A\kern-.1667em\lower.5ex\hbox{M}\kern-.125emS}}
\providecommand*{\eTeX}{\ensuremath{\varepsilon}-\TeX}
\DeclareRobustCommand*{\XeTeX}
  {X\kern-.125em\lower.5ex\hbox{\reflectbox{E}}\kern-.1667em\TeX}
\providecommand*{\CTAN}{\textsc{ctan}}
\patchcmd{\@addmarginpar}
  {\@latex@warning@no@line {Marginpar on page \thepage\space moved}}
  {\relax}{}{}
\newcounter{argument}
\g@addto@macro\endmacro{\setcounter{argument}{0}}
\newcommand*\darg[1]{%
  \stepcounter{argument}%
  {\ttfamily\char`\#\theargument~:~}#1\par\noindent\ignorespaces}
\newcommand*\doarg[1]{%
  \stepcounter{argument}%
  {\ttfamily\makebox[0pt][r]{[}%
   \char`\#\theargument]:~}#1\par\noindent\ignorespaces}
%</jawltxdoc>
%\fi

%
% Documentation:
%    (a) Without write18 enabled:
%          pdflatex pgfopts.dtx
%          makeindex -s gind.ist pgfopts.idx
%          makeindex -s gglo.ist -o pgfopts.gls  pgfopts.glo
%          pdflatex pgfopts.dtx
%          pdflatex pgfopts.dtx
%    (b) With write18 enabled:
%          pdflatex pgfopts.dtx
%          pdflatex pgfopts.dtx
%          pdflatex pgfopts.dtx
%
% Installation:
%     Copy pgfopts.sty to a location searched by TeX, and if required
%     by your TeX installation, run the appropriate command to build
%     a hash of files (texhash, initexmf --update-fndb, etc.)
%
% Note:
%     The jawltxdoc.sty file is not needed for installation,
%     only for building the documentation; it may be deleted
%     after producing the documentation (if necessary).
%
%<*ignore>
% This is all taken verbatim from Heiko Oberdiek's packages
\begingroup
  \def\x{LaTeX2e}%
\expandafter\endgroup
\ifcase 0\ifx\install y1\fi\expandafter
         \ifx\csname processbatchFile\endcsname\relax\else1\fi
         \ifx\fmtname\x\else 1\fi\relax
\else\csname fi\endcsname
%</ignore>
%<*install>
\input docstrip.tex
\keepsilent
\askforoverwritefalse
\preamble
 --------------------------------------------------------------
 The pgfopts package --- LaTeX package options with pgfkeys
 Maintained by Joseph Wright
 E-mail: joseph.wright@morningstar2.co.uk
 Released under the LaTeX Project Public License v1.3 or later
 See http://www.latex-project.org/lppl.txt
 --------------------------------------------------------------

\endpreamble
\Msg{Generating pgfopts files:}
\generate{\file{jawltxdoc.sty}{\from{\jobname.dtx}{jawltxdoc}}
}
\usedir{tex/latex/pgfopts}
\generate{\file{\jobname.sty}{\from{\jobname.dtx}{package}}
}
\usedir{source/latex/pgfopts}
\generate{\file{\jobname.ins}{\from{\jobname.dtx}{install}}
}
\nopreamble\nopostamble
\usedir{doc/latex/pgfopts}
\generate{\file{README.txt}{\from{\jobname.dtx}{readme}}
}
\endbatchfile
%</install>
%<*readme>
---------------------------------------------------------------
The pgfopts package --- LaTeX package options with pgfkeys
Maintained by Joseph Wright
E-mail: joseph.wright@morningstar2.co.uk
Released under the LaTeX Project Public License v1.3c or later
See http://www.latex-project.org/lppl.txt
---------------------------------------------------------------

Using key-value options for packages and macros is a good way
of handling large numbers of options with a clean interface.
The pgfkeys package provides a very well designed system for
defining and using keys, but does not make this available for
handling LaTeX class and package options.  The pgfopts package
adds this ability to pgfkeys, in the same way that kvoptions
extends the keyval package.

%</readme>
%<*ignore>
\fi
% Will Robertson's trick
\immediate\write18{makeindex -s gind.ist -o \jobname.ind  \jobname.idx}
\immediate\write18{makeindex -s gglo.ist -o \jobname.gls  \jobname.glo}
%</ignore>
%<*driver>
\PassOptionsToClass{a4paper}{article}
\documentclass[german,english,UKenglish]{ltxdoc}
\EnableCrossrefs
\CodelineIndex
\RecordChanges
%\OnlyDescription
\usepackage{jawltxdoc}
\begin{document}
  \DocInput{\jobname.dtx}
\end{document}
%</driver>
% \fi
%
%\CheckSum{220}
%
% \CharacterTable
%  {Upper-case    \A\B\C\D\E\F\G\H\I\J\K\L\M\N\O\P\Q\R\S\T\U\V\W\X\Y\Z
%   Lower-case    \a\b\c\d\e\f\g\h\i\j\k\l\m\n\o\p\q\r\s\t\u\v\w\x\y\z
%   Digits        \0\1\2\3\4\5\6\7\8\9
%   Exclamation   \!     Double quote  \"     Hash (number) \#
%   Dollar        \$     Percent       \%     Ampersand     \&
%   Acute accent  \'     Left paren    \(     Right paren   \)
%   Asterisk      \*     Plus          \+     Comma         \,
%   Minus         \-     Point         \.     Solidus       \/
%   Colon         \:     Semicolon     \;     Less than     \<
%   Equals        \=     Greater than  \>     Question mark \?
%   Commercial at \@     Left bracket  \[     Backslash     \\
%   Right bracket \]     Circumflex    \^     Underscore    \_
%   Grave accent  \`     Left brace    \{     Vertical bar  \|
%   Right brace   \}     Tilde         \~}
%
% \GetFileInfo{\jobname.sty}
%
%\changes{v1.0}{2008/06/26}{First public release}
%
%\setkeys{lst}{language=[LaTeX]{TeX},moretexcs={pgfkeys,
%  ProcessPgfOptions,pgfkeyscurrentname,pgfkeyscurrentvalue}}
%
%\DoNotIndex{\.,\/,\:,\=,\@classoptionslist,\@clsextension}
%\DoNotIndex{\@currext,\@currname,\@empty,\@expandtwoargs,\@for}
%\DoNotIndex{\@ifstar,\@ifundefined,\@makeother,\@nil,\@onlypreamble}
%\DoNotIndex{\@removeelement,\@tempa,\@unprocessedoptions}
%\DoNotIndex{\@unusedoptionlist,\AtEndOfPackage,\begingroup}
%\DoNotIndex{\catcode,\csname,\CurrentOption,\def,\do,\edef}
%\DoNotIndex{\else,\endcsname,\endgroup,\expandafter,\fi}
%\DoNotIndex{\gdef,\global,\ifx,\let,\newcommand,\noexpand}
%\DoNotIndex{\NeedsTeXFormat,\pgfkeys,\pgfkeysifdefined}
%\DoNotIndex{\ProvidesPackage,\relax,\RequirePackage,\space}
%\DoNotIndex{\string,\the,\toks@}
%
% \title{\currpkg\ --- \LaTeX\ package options with pgfkeys^^A
%   \thanks{This file describes version \fileversion, last revised
%           \filedate.}}
% \author{Joseph Wright^^A
%   \thanks{E-mail: joseph.wright@morningstar2.co.uk}}
% \date{Released \filedate}
%
% \maketitle
%
%\begin{abstract}
% Using key--value options for packages and macros is a good way of
% handling large numbers of options with a clean interface. The
% \pkg{pgfkeys} package provides a very well designed system for
% defining and using keys, but does not make this available for
% handling \LaTeX\ class and package options.  The \currpkg package
% adds this ability to \pkg{pgfkeys}, in the same way that
% \pkg{kvoptions} extends the \pkg{keyval} package.
%\end{abstract}
%
%\begin{multicols}{2}
%\tableofcontents
%\end{multicols}
%
%\section{Introduction}
% The key--value method for optional arguments is very popular, as it
% allows the class or package author to define a large number of
% options with a simple interface.  A number of packages can be used
% to provide the key management system, most of which load or extent
% the parent \pkg{keyval} package.  On its own, \pkg{keyval} can only
% be used for parsing the arguments of macros.  However, a number of
% packages have extended the method to processing \LaTeX\ class and
% package options.  This processing is made available as part of two
% general-purpose packages \pkg{xkeyval} and \pkg{kvoptions}; both
% allow the author of a class or package to process key--value
% options given at load-time.
%
% The \pkg{pgfkeys} package provides a somewhat different key--value
% system to \pkg{keyval} and derivatives.  This uses a completely
% different model for defining and using keys, although for the
% end-user the result appears very similar.  The \currpkg package
% allows keys defined with \pkg{pgfkeys} to be used as class or
% package options, in the same way that \pkg{kvoptions} extends
% \pkg{keyval}.
%
% Users of \currpkg should be familiar with the general methods used
% by \pkg{pgfkeys}.  These are outlined in the manual for the
% Ti\emph{k}z and \pkg{pgf} bundle.
%
%\section{Using the package}
%\subsection{Creating options}
% To create package or class options for use with \currpkg, it is
% only necessary to define the appropriate keys.  Taking as an
% example a package ``\pkg{MyOwnPackage}'', which uses the prefix
% \texttt{MOP} on internal macros, creating keys would take the form:
%\begin{LaTeXexample}[noexample]
%  \pgfkeys{/MOP/.cd,
%    keyone/.code=\wlog{Value `#1' given},
%    keytwo/.store in=\MOP@store}
%\end{LaTeXexample}
% Here, \opt{keyone} simply writes its argument to the log, while
% \opt{keytwo} stores the value given in the \cs{MOP@store} macro.
%
% An important point to notice is that the key names \emph{do not}
% contain a space. This is because the \LaTeX\ kernel removes spaces
% from options before they are passed to the class or package. Spaces
% can occur in the path to the key, but not in the key name itself.
% This restriction only applies to keys used as options.
%
%\subsection{Processing options}
%\DescribeMacro{\ProcessPgfOptions}
% Options should be processed using the \cs{ProcessPgfOptions} macro,
% which takes two forms.  When invoked followed by a star, the system
% will use the name of the current file as the key path.  Thus, for
% the example \pkg{MyOwnPackage.sty}, \currpkg would use the path
% \opt{/MyOwnPackage}.
%\begin{LaTeXexample}[noexample]
%  \ProcessPgfOptions*
%\end{LaTeXexample}
% When the key path does not match the current file name, the second
% form of \cs{ProcessPgfOptions} should be used.  The full path to be
% searched should be given as the argument to the macro.  Continuing
% the example, for \pkg{MyOwnPackage.sty} using path \opt{/MOP}, the
% correct call is:
%\begin{LaTeXexample}[noexample]
%  \ProcessPgfOptions{/MOP}
%\end{LaTeXexample}
% Notice that the \emph{full} path is given here, including the
% leading ``|/|''; this allows the advance user to alter the default
% path before calling \cs{ProcessPgfOptions}.
%
%\subsection{Specialised options}
% Users of \pkg{kvoptions} will be aware that there are two types of
% options which need specialised handling.  First, there is the need
% to process options which should not be given a value. Using
% \pkg{pgfkeys}, this is achieved using the \opt{.value forbidden}
% handler.
%\begin{LaTeXexample}[noexample]
%  \pgfkeys{/MOP/.cd,
%    keythree/.code=\wlog{keythree set},
%    keythree/.value forbidden}
%\end{LaTeXexample}
%
% The second type of specialist option is the unknown option.
% Some packages accept arbitrary values in the options, which
% can then be used to load configuration files, \etc.  Once
% again, this is handled using correctly-defined keys.
%\begin{LaTeXexample}[noexample]
%  \pgfkeys{/MOP/.cd,
%    .unknown/.code=\wlog{%
%      The unknown key `\pgfkeyscurrentname' was given with
%      value `\pgfkeyscurrentvalue'}}
%\end{LaTeXexample}
%
% The power of the \pkg{pgfkeys} system is demonstrated here: no
% additional macros are needed to handle the specialised option
% types.
%
% \StopEventually{^^A
%   \PrintChanges
%   \PrintIndex}
%
% \iffalse
%<*package>
% \fi
%
% \section{Implementation}
% The code here is based heavily on \pkg{kvoptions}, which has a
% similar aim to \currpkg, but works with \pkg{keyval} and derived
% packages.
%
%\begin{macro}{\pgfopts@id}
% Macro naming follows the scheme used by \pkg{pgfkeys}: the full
% package name is the prefix.
%    \begin{macrocode}
\NeedsTeXFormat{LaTeX2e}
\newcommand*{\pgfopts@id}{}
\def\pgfopts@id$#1: #2.#3 #4 #5-#6-#7 #8 #9${%
    #5/#6/#7\space v1.0\space}
\ProvidesPackage{pgfopts}
  [\pgfopts@id $Id: pgfopts.dtx 30 2008-06-26 19:27:20Z joseph $
   LaTeX package options with pgfkeys]
%    \end{macrocode}
%\end{macro}
% The \pkg{pgfkeys} package is needed, unsurprisingly.
%    \begin{macrocode}
\RequirePackage{pgfkeys}
%    \end{macrocode}
%
%\begin{macro}{\pgfopts@catcodes}
% A few category codes are set correctly, to prevent odd
% accidents.
%    \begin{macrocode}
\edef\pgfopts@catcodes{%
  \catcode\string`\string ` \the\catcode\string`\`\relax
  \catcode\string`\string = \the\catcode\string`\=\relax
  \catcode\string`\string : \the\catcode\string`\:\relax
  \catcode\string`\string , \the\catcode\string`\,\relax
  \catcode\string`\string / \the\catcode\string`\/\relax
  \catcode\string`\string . \the\catcode\string`\.\relax}
\catcode\string`\` 12\relax
\catcode`\= 12\relax
\@makeother{\:}
\@makeother{\,}
\@makeother{\/}
\@makeother{\.}
%    \end{macrocode}
%\end{macro}
%
%\begin{macro}{\ProcessPgfOptions}
% The user macro for processing options may have a star.  In that
% case, \cs{@currname} is used for setting keys.
%    \begin{macrocode}
\newcommand*{\ProcessPgfOptions}{%
  \@ifstar
    {\begingroup
       \edef\@tempa{%
         \endgroup
         \noexpand\pgfopts@ProcessOptions
           {/\@currname}}%
     \@tempa}%
    {\pgfopts@ProcessOptions}}
\@onlypreamble\ProcessPgfOptions
%    \end{macrocode}
%\end{macro}
%\begin{macro}{\pgfopts@temp}
% A private storage area is needed by the next macro.
%    \begin{macrocode}
\newcommand*{\pgfopts@temp}{}
%    \end{macrocode}
%\end{macro}
%\begin{macro}{\pgfopts@ProcessOptions}
%  \darg{path}
% The business end of the package.  This is a modified copy of the
% macro \cs{KVO@ProcessOptions} from \pkg{kvoptions}.  Some
% simplifications are made due to the differences in key system, and
% as patching the kernel is ignored.
%    \begin{macrocode}
\newcommand*{\pgfopts@ProcessOptions}[1]{%
  \def\pgfopts@temp{,}%
  \ifx\@currext\@clsextension\else
    \ifx\@classoptionslist\relax\else
%    \end{macrocode}
%\begin{macro}{\pgfopts@CurrentOption}
% For packages, the list of global (class) options needs to be
% checked. The assumption is made that only \emph{explicitly}
% declared keys will be accepted as options here. The strange
% argument to \cs{pgfkeysifdefined} is needed as checking for the key
% itself only works if it has already been used.
%    \begin{macrocode}
      \@for\pgfopts@CurrentOption:=\@classoptionslist\do{%
        \pgfkeysifdefined{#1/\expandafter\pgfopts@sepkeyval%
          \pgfopts@CurrentOption=\@nil/.@cmd}
          {\edef\pgfopts@temp{%
             \pgfopts@temp\pgfopts@CurrentOption,}%
           \@expandtwoargs\@removeelement\pgfopts@CurrentOption
             \@unusedoptionlist\@unusedoptionlist}%
          {}}%
    \fi
  \fi
%    \end{macrocode}
%\end{macro}
% Now the local options have to be processed.  If there are none,
% then the global list is simply copied.  Otherwise, the local
% options are first added to the ``to do'' list stored in \cs{toks@}.
%    \begin{macrocode}
  \begingroup
    \@ifundefined{opt@\@currname.\@currext}
      {\toks@\expandafter{\pgfopts@temp}}
      {\toks@\expandafter\expandafter\expandafter{%
         \csname opt@\@currname.\@currext\endcsname}%
%    \end{macrocode}
% There is work to do if processing options of a class; unknown options
% from the global list have to be completely ignored.
%    \begin{macrocode}
       \ifx\@currext\@clsextension
         \edef\@tempa{\the\toks@}%
         \toks@\expandafter{\pgfopts@temp}%
         \@for\CurrentOption:=\@tempa\do{%
           \pgfkeysifdefined{#1/\expandafter%
             \pgfopts@sepkeyval\CurrentOption=\@nil/.@cmd}
             {\toks@\expandafter{\the\expandafter\toks@
                \expandafter,\CurrentOption}}
             {\ifx\@empty\@unusedoptionlist\@empty
                \global\let\@unusedoptionlist\CurrentOption
              \else
                \expandafter\expandafter\expandafter\gdef
                \expandafter\expandafter\expandafter
                \@unusedoptionlist\expandafter\expandafter
                \expandafter{%
                  \expandafter\@unusedoptionlist
                  \expandafter,\CurrentOption}%
              \fi}}%
%    \end{macrocode}
% For packages, the two lists are simply combined.
%    \begin{macrocode}
      \else
        \toks@\expandafter\expandafter\expandafter{%
            \expandafter\pgfopts@temp\the\toks@}%
      \fi}%
%    \end{macrocode}
% With everything done, the keys are sent to \pkg{pgfkeys} for
% processing.  The default key system in \pkg{pgfkeys} is very
% different from that in \pkg{kvoptions}, and so the system here is
% much less complex.
%    \begin{macrocode}
    \edef\pgfopts@temp{\endgroup
      \noexpand\pgfkeys{#1/.cd,\the\toks@}}%
  \pgfopts@temp
  \let\CurrentOption\@empty
  \AtEndOfPackage{\let\@unprocessedoptions\relax}}
%    \end{macrocode}
%\end{macro}
%\begin{macro}{\pgfopts@sepkeyval}
% A very simply macro to separate a key from a potential value.
%    \begin{macrocode}
\newcommand*{\pgfopts@sepkeyval}{}
\def\pgfopts@sepkeyval#1=#2\@nil{#1}
%    \end{macrocode}
%\end{macro}
% Tidying up.
%    \begin{macrocode}
\pgfopts@catcodes
%    \end{macrocode}
%
%\Finale
%
%\iffalse
%</package>
%<*jawltxdoc>
\NeedsTeXFormat{LaTeX2e}
\ProvidesPackage{jawltxdoc}
\usepackage[T1]{fontenc}
\usepackage{lmodern}
\usepackage[final]{listings,graphicx,microtype}
\usepackage[scaled=0.95]{helvet}
\usepackage[version=3]{mhchem}
\usepackage[osf]{mathpazo}
\usepackage{booktabs,array,url,courier,xspace,etoolbox,varioref}
\usepackage{upgreek,ifpdf,float,caption,longtable,babel}
\floatstyle{plaintop}
\restylefloat{table}
\labelformat{figure}{\figurename~#1}
\labelformat{table}{\tablename~#1}
\ifpdf
  \usepackage{embedfile}
  \embedfile[%
    stringmethod=escape,%
    mimetype=plain/text,%
    desc={LaTeX docstrip source archive for package `\jobname'}%
    ]{\jobname.dtx}
\fi
\IfFileExists{\jobname.sty}
  {\usepackage{\jobname}}{}
\usepackage[numbered]{hypdoc}
\setcounter{IndexColumns}{2}
\newlength\LaTeXwidth
\newlength\LaTeXoutdent
\newlength\LaTeXgap
\setlength\LaTeXgap{1em}
\setlength\LaTeXoutdent{-0.15\textwidth}
\newbox\lst@samplebox
\edef\LaTeXexamplefile{\jobname.tmp}
\lst@RequireAspects{writefile}
\lstnewenvironment{LaTeXexample}[1][example]{%
  \global\let\lst@intname\@empty
  \ifcsname LaTeXcode#1\endcsname
    \expandafter\let\expandafter\LaTeXcode
      \csname LaTeXcode#1\endcsname
    \expandafter\let\expandafter\LaTeXcodeend
      \csname LaTeXcode#1end\endcsname
  \else
    \PackageError{jawltxdoc}
      {Undefined example type `#1'}
      \@ehd
    \let\LaTeXcode\relax
    \let\LaTeXcodeend\relax
  \fi
  \LaTeXcode}
  {\lst@EndWriteFile
   \LaTeXcodeend}
\newcommand*{\LaTeXcodeexample}{%
  \setbox\lst@samplebox=\hbox\bgroup
  \LaTeXcodefloat}
\let\LaTeXcoderesultonly\LaTeXcodeexample
\newcommand*{\LaTeXcodeexampleend}{%
  \egroup
  \setlength\LaTeXwidth{\wd\lst@samplebox}%
  \begin{list}{}{%
    \setlength\itemindent{0pt}
    \setlength\leftmargin\LaTeXoutdent
    \setlength\rightmargin{0pt}}%
    \item
      \setlength\LaTeXoutdent{-0.15\textwidth}
      \begin{minipage}[c]{%
        \textwidth-\LaTeXwidth-\LaTeXoutdent-\LaTeXgap}
        \LaTeXcodefloatend
      \end{minipage}%
      \hfill
      \begin{minipage}[c]{\LaTeXwidth}%
        \hbox to\linewidth{\box\lst@samplebox\hss}%
      \end{minipage}%
  \end{list}}
\let\LaTeXcodenoexample\LaTeXcodeexample
\newcommand*{\LaTeXcodenoexampleend}{%
  \egroup
  \setlength\LaTeXwidth{\wd\lst@samplebox}%
  \begin{list}{}{%
    \setlength\itemindent{0pt}
    \setlength\leftmargin{0pt}
    \setlength\rightmargin{0pt}}%
    \item
      \begin{minipage}[c]{\LaTeXwidth}
        \hbox to\linewidth{\box\lst@samplebox\hss}%
      \end{minipage}%
  \end{list}}
\newcommand*{\LaTeXcodefloat}{%
  \setkeys{lst}{tabsize=4,gobble=3,breakindent=0pt,
    basicstyle=\small\ttfamily,basewidth=0.51em,
    keywordstyle=\color{blue}}%
  \lst@BeginAlsoWriteFile{\LaTeXexamplefile}}
\newcommand*{\LaTeXcodefloatend}{%
  \MakePercentComment\catcode`\^^M=10\relax
  \small
  {\setkeys{lst}{SelectCharTable=\lst@ReplaceInput{\^\^I}%
    {\lst@ProcessTabulator}}%
    \leavevmode \input{\LaTeXexamplefile}}%
  \MakePercentIgnore}
\newcommand*{\LaTeXcoderesultonlyend}{\egroup\LaTeXcodefloatend}
\lstnewenvironment{BibTeXexample}{%
  \global\let\lst@intname\@empty
  \setbox\lst@samplebox=\hbox\bgroup
  \setkeys{lst}{tabsize=4,gobble=3,breakindent=0pt,
    basicstyle=\small\ttfamily,basewidth=0.51em,
    keywordstyle=\color{black}}
  \lst@BeginAlsoWriteFile{\LaTeXexamplefile}}
 {\lst@EndWriteFile
   \LaTeXcodeexampleend}
\newcommand*{\DescribeOption}{%
  \leavevmode\@bsphack\begingroup\MakePrivateLetters
  \Describe@Option}
\newcommand*{\Describe@Option}[1]{\endgroup
              \marginpar{\raggedleft\PrintDescribeEnv{#1}}%
              \SpecialOptionIndex{#1}\@esphack\ignorespaces}
\newcommand*{\SpecialOptionIndex}[1]{\@bsphack
    \index{#1\actualchar{\protect\ttfamily#1}
           (option)\encapchar usage}%
    \index{options:\levelchar#1\actualchar{\protect\ttfamily#1}%
      \encapchar usage}\@esphack}
\newcommand*{\indexopt}[1]{\DescribeOption{#1}\opt{#1}}
\newcommand*{\DescribeOptionInfo}[2]{%
  \DescribeOption{#1}%
  \opt{#1=\meta{#2}}\xspace}
\newcommand*{\ofixarg}[1]{%
  {\ttfamily[}%
  \ifmmode \expandafter \nfss@text \fi
  {%
    \meta@font@select
    \edef\meta@hyphen@restore{%
      \hyphenchar\the\font\the\hyphenchar\font}%
    \hyphenchar\font\m@ne
    \language\l@nohyphenation
    #1\/%
    \meta@hyphen@restore
    }%
    {\ttfamily]}}
\newcommand*{\pkg}[1]{\textsf{#1}}
\newcommand*{\currpkg}{\pkg{\jobname}\xspace}
\newcommand*{\opt}[1]{\texttt{#1}}
\newcommand*{\defaultopt}[1]{\opt{\textbf{#1}}}
\newcommand*{\file}[1]{\texttt{#1}}
\newcommand*{\ext}[1]{\file{.#1}}
\newcommand*{\latin}[1]{\emph{#1}}
\newcommand*{\etc}{%
  \@ifnextchar.
    {\latin{etc}}
    {\latin{etc}.\xspace}}
\newcommand*{\eg}{%
  \@ifnextchar.
    {\latin{e.g}}
    {\latin{e.g}.\xspace}}
\newcommand*{\ie}{%
  \@ifnextchar.
    {\latin{i.e}}
    {\latin{i.e}.\xspace}}
\newcommand*{\etal}{%
  \@ifnextchar.
    {\latin{et~al.}}
    {\latin{et~al}.\xspace}}
\newcommand*{\AMS}{{\protect\usefont{OMS}{cmsy}{m}{n}%
  A\kern-.1667em\lower.5ex\hbox{M}\kern-.125emS}}
\providecommand*{\eTeX}{\ensuremath{\varepsilon}-\TeX}
\DeclareRobustCommand*{\XeTeX}
  {X\kern-.125em\lower.5ex\hbox{\reflectbox{E}}\kern-.1667em\TeX}
\providecommand*{\CTAN}{\textsc{ctan}}
\patchcmd{\@addmarginpar}
  {\@latex@warning@no@line {Marginpar on page \thepage\space moved}}
  {\relax}{}{}
\newcounter{argument}
\g@addto@macro\endmacro{\setcounter{argument}{0}}
\newcommand*\darg[1]{%
  \stepcounter{argument}%
  {\ttfamily\char`\#\theargument~:~}#1\par\noindent\ignorespaces}
\newcommand*\doarg[1]{%
  \stepcounter{argument}%
  {\ttfamily\makebox[0pt][r]{[}%
   \char`\#\theargument]:~}#1\par\noindent\ignorespaces}
%</jawltxdoc>
%\fi

%
% Documentation:
%    (a) Without write18 enabled:
%          pdflatex pgfopts.dtx
%          makeindex -s gind.ist pgfopts.idx
%          makeindex -s gglo.ist -o pgfopts.gls  pgfopts.glo
%          pdflatex pgfopts.dtx
%          pdflatex pgfopts.dtx
%    (b) With write18 enabled:
%          pdflatex pgfopts.dtx
%          pdflatex pgfopts.dtx
%          pdflatex pgfopts.dtx
%
% Installation:
%     Copy pgfopts.sty to a location searched by TeX, and if required
%     by your TeX installation, run the appropriate command to build
%     a hash of files (texhash, initexmf --update-fndb, etc.)
%
% Note:
%     The jawltxdoc.sty file is not needed for installation,
%     only for building the documentation; it may be deleted
%     after producing the documentation (if necessary).
%
%<*ignore>
% This is all taken verbatim from Heiko Oberdiek's packages
\begingroup
  \def\x{LaTeX2e}%
\expandafter\endgroup
\ifcase 0\ifx\install y1\fi\expandafter
         \ifx\csname processbatchFile\endcsname\relax\else1\fi
         \ifx\fmtname\x\else 1\fi\relax
\else\csname fi\endcsname
%</ignore>
%<*install>
\input docstrip.tex
\keepsilent
\askforoverwritefalse
\preamble
 --------------------------------------------------------------
 The pgfopts package --- LaTeX package options with pgfkeys
 Maintained by Joseph Wright
 E-mail: joseph.wright@morningstar2.co.uk
 Released under the LaTeX Project Public License v1.3 or later
 See http://www.latex-project.org/lppl.txt
 --------------------------------------------------------------

\endpreamble
\Msg{Generating pgfopts files:}
\generate{\file{jawltxdoc.sty}{\from{\jobname.dtx}{jawltxdoc}}
}
\usedir{tex/latex/pgfopts}
\generate{\file{\jobname.sty}{\from{\jobname.dtx}{package}}
}
\usedir{source/latex/pgfopts}
\generate{\file{\jobname.ins}{\from{\jobname.dtx}{install}}
}
\nopreamble\nopostamble
\usedir{doc/latex/pgfopts}
\generate{\file{README.txt}{\from{\jobname.dtx}{readme}}
}
\endbatchfile
%</install>
%<*readme>
---------------------------------------------------------------
The pgfopts package --- LaTeX package options with pgfkeys
Maintained by Joseph Wright
E-mail: joseph.wright@morningstar2.co.uk
Released under the LaTeX Project Public License v1.3c or later
See http://www.latex-project.org/lppl.txt
---------------------------------------------------------------

Using key-value options for packages and macros is a good way
of handling large numbers of options with a clean interface.
The pgfkeys package provides a very well designed system for
defining and using keys, but does not make this available for
handling LaTeX class and package options.  The pgfopts package
adds this ability to pgfkeys, in the same way that kvoptions
extends the keyval package.

%</readme>
%<*ignore>
\fi
% Will Robertson's trick
\immediate\write18{makeindex -s gind.ist -o \jobname.ind  \jobname.idx}
\immediate\write18{makeindex -s gglo.ist -o \jobname.gls  \jobname.glo}
%</ignore>
%<*driver>
\PassOptionsToClass{a4paper}{article}
\documentclass[german,english,UKenglish]{ltxdoc}
\EnableCrossrefs
\CodelineIndex
\RecordChanges
%\OnlyDescription
\usepackage{jawltxdoc}
\begin{document}
  \DocInput{\jobname.dtx}
\end{document}
%</driver>
% \fi
%
%\CheckSum{220}
%
% \CharacterTable
%  {Upper-case    \A\B\C\D\E\F\G\H\I\J\K\L\M\N\O\P\Q\R\S\T\U\V\W\X\Y\Z
%   Lower-case    \a\b\c\d\e\f\g\h\i\j\k\l\m\n\o\p\q\r\s\t\u\v\w\x\y\z
%   Digits        \0\1\2\3\4\5\6\7\8\9
%   Exclamation   \!     Double quote  \"     Hash (number) \#
%   Dollar        \$     Percent       \%     Ampersand     \&
%   Acute accent  \'     Left paren    \(     Right paren   \)
%   Asterisk      \*     Plus          \+     Comma         \,
%   Minus         \-     Point         \.     Solidus       \/
%   Colon         \:     Semicolon     \;     Less than     \<
%   Equals        \=     Greater than  \>     Question mark \?
%   Commercial at \@     Left bracket  \[     Backslash     \\
%   Right bracket \]     Circumflex    \^     Underscore    \_
%   Grave accent  \`     Left brace    \{     Vertical bar  \|
%   Right brace   \}     Tilde         \~}
%
% \GetFileInfo{\jobname.sty}
%
%\changes{v1.0}{2008/06/26}{First public release}
%
%\setkeys{lst}{language=[LaTeX]{TeX},moretexcs={pgfkeys,
%  ProcessPgfOptions,pgfkeyscurrentname,pgfkeyscurrentvalue}}
%
%\DoNotIndex{\.,\/,\:,\=,\@classoptionslist,\@clsextension}
%\DoNotIndex{\@currext,\@currname,\@empty,\@expandtwoargs,\@for}
%\DoNotIndex{\@ifstar,\@ifundefined,\@makeother,\@nil,\@onlypreamble}
%\DoNotIndex{\@removeelement,\@tempa,\@unprocessedoptions}
%\DoNotIndex{\@unusedoptionlist,\AtEndOfPackage,\begingroup}
%\DoNotIndex{\catcode,\csname,\CurrentOption,\def,\do,\edef}
%\DoNotIndex{\else,\endcsname,\endgroup,\expandafter,\fi}
%\DoNotIndex{\gdef,\global,\ifx,\let,\newcommand,\noexpand}
%\DoNotIndex{\NeedsTeXFormat,\pgfkeys,\pgfkeysifdefined}
%\DoNotIndex{\ProvidesPackage,\relax,\RequirePackage,\space}
%\DoNotIndex{\string,\the,\toks@}
%
% \title{\currpkg\ --- \LaTeX\ package options with pgfkeys^^A
%   \thanks{This file describes version \fileversion, last revised
%           \filedate.}}
% \author{Joseph Wright^^A
%   \thanks{E-mail: joseph.wright@morningstar2.co.uk}}
% \date{Released \filedate}
%
% \maketitle
%
%\begin{abstract}
% Using key--value options for packages and macros is a good way of
% handling large numbers of options with a clean interface. The
% \pkg{pgfkeys} package provides a very well designed system for
% defining and using keys, but does not make this available for
% handling \LaTeX\ class and package options.  The \currpkg package
% adds this ability to \pkg{pgfkeys}, in the same way that
% \pkg{kvoptions} extends the \pkg{keyval} package.
%\end{abstract}
%
%\begin{multicols}{2}
%\tableofcontents
%\end{multicols}
%
%\section{Introduction}
% The key--value method for optional arguments is very popular, as it
% allows the class or package author to define a large number of
% options with a simple interface.  A number of packages can be used
% to provide the key management system, most of which load or extent
% the parent \pkg{keyval} package.  On its own, \pkg{keyval} can only
% be used for parsing the arguments of macros.  However, a number of
% packages have extended the method to processing \LaTeX\ class and
% package options.  This processing is made available as part of two
% general-purpose packages \pkg{xkeyval} and \pkg{kvoptions}; both
% allow the author of a class or package to process key--value
% options given at load-time.
%
% The \pkg{pgfkeys} package provides a somewhat different key--value
% system to \pkg{keyval} and derivatives.  This uses a completely
% different model for defining and using keys, although for the
% end-user the result appears very similar.  The \currpkg package
% allows keys defined with \pkg{pgfkeys} to be used as class or
% package options, in the same way that \pkg{kvoptions} extends
% \pkg{keyval}.
%
% Users of \currpkg should be familiar with the general methods used
% by \pkg{pgfkeys}.  These are outlined in the manual for the
% Ti\emph{k}z and \pkg{pgf} bundle.
%
%\section{Using the package}
%\subsection{Creating options}
% To create package or class options for use with \currpkg, it is
% only necessary to define the appropriate keys.  Taking as an
% example a package ``\pkg{MyOwnPackage}'', which uses the prefix
% \texttt{MOP} on internal macros, creating keys would take the form:
%\begin{LaTeXexample}[noexample]
%  \pgfkeys{/MOP/.cd,
%    keyone/.code=\wlog{Value `#1' given},
%    keytwo/.store in=\MOP@store}
%\end{LaTeXexample}
% Here, \opt{keyone} simply writes its argument to the log, while
% \opt{keytwo} stores the value given in the \cs{MOP@store} macro.
%
% An important point to notice is that the key names \emph{do not}
% contain a space. This is because the \LaTeX\ kernel removes spaces
% from options before they are passed to the class or package. Spaces
% can occur in the path to the key, but not in the key name itself.
% This restriction only applies to keys used as options.
%
%\subsection{Processing options}
%\DescribeMacro{\ProcessPgfOptions}
% Options should be processed using the \cs{ProcessPgfOptions} macro,
% which takes two forms.  When invoked followed by a star, the system
% will use the name of the current file as the key path.  Thus, for
% the example \pkg{MyOwnPackage.sty}, \currpkg would use the path
% \opt{/MyOwnPackage}.
%\begin{LaTeXexample}[noexample]
%  \ProcessPgfOptions*
%\end{LaTeXexample}
% When the key path does not match the current file name, the second
% form of \cs{ProcessPgfOptions} should be used.  The full path to be
% searched should be given as the argument to the macro.  Continuing
% the example, for \pkg{MyOwnPackage.sty} using path \opt{/MOP}, the
% correct call is:
%\begin{LaTeXexample}[noexample]
%  \ProcessPgfOptions{/MOP}
%\end{LaTeXexample}
% Notice that the \emph{full} path is given here, including the
% leading ``|/|''; this allows the advance user to alter the default
% path before calling \cs{ProcessPgfOptions}.
%
%\subsection{Specialised options}
% Users of \pkg{kvoptions} will be aware that there are two types of
% options which need specialised handling.  First, there is the need
% to process options which should not be given a value. Using
% \pkg{pgfkeys}, this is achieved using the \opt{.value forbidden}
% handler.
%\begin{LaTeXexample}[noexample]
%  \pgfkeys{/MOP/.cd,
%    keythree/.code=\wlog{keythree set},
%    keythree/.value forbidden}
%\end{LaTeXexample}
%
% The second type of specialist option is the unknown option.
% Some packages accept arbitrary values in the options, which
% can then be used to load configuration files, \etc.  Once
% again, this is handled using correctly-defined keys.
%\begin{LaTeXexample}[noexample]
%  \pgfkeys{/MOP/.cd,
%    .unknown/.code=\wlog{%
%      The unknown key `\pgfkeyscurrentname' was given with
%      value `\pgfkeyscurrentvalue'}}
%\end{LaTeXexample}
%
% The power of the \pkg{pgfkeys} system is demonstrated here: no
% additional macros are needed to handle the specialised option
% types.
%
% \StopEventually{^^A
%   \PrintChanges
%   \PrintIndex}
%
% \iffalse
%<*package>
% \fi
%
% \section{Implementation}
% The code here is based heavily on \pkg{kvoptions}, which has a
% similar aim to \currpkg, but works with \pkg{keyval} and derived
% packages.
%
%\begin{macro}{\pgfopts@id}
% Macro naming follows the scheme used by \pkg{pgfkeys}: the full
% package name is the prefix.
%    \begin{macrocode}
\NeedsTeXFormat{LaTeX2e}
\newcommand*{\pgfopts@id}{}
\def\pgfopts@id$#1: #2.#3 #4 #5-#6-#7 #8 #9${%
    #5/#6/#7\space v1.0\space}
\ProvidesPackage{pgfopts}
  [\pgfopts@id $Id: pgfopts.dtx 30 2008-06-26 19:27:20Z joseph $
   LaTeX package options with pgfkeys]
%    \end{macrocode}
%\end{macro}
% The \pkg{pgfkeys} package is needed, unsurprisingly.
%    \begin{macrocode}
\RequirePackage{pgfkeys}
%    \end{macrocode}
%
%\begin{macro}{\pgfopts@catcodes}
% A few category codes are set correctly, to prevent odd
% accidents.
%    \begin{macrocode}
\edef\pgfopts@catcodes{%
  \catcode\string`\string ` \the\catcode\string`\`\relax
  \catcode\string`\string = \the\catcode\string`\=\relax
  \catcode\string`\string : \the\catcode\string`\:\relax
  \catcode\string`\string , \the\catcode\string`\,\relax
  \catcode\string`\string / \the\catcode\string`\/\relax
  \catcode\string`\string . \the\catcode\string`\.\relax}
\catcode\string`\` 12\relax
\catcode`\= 12\relax
\@makeother{\:}
\@makeother{\,}
\@makeother{\/}
\@makeother{\.}
%    \end{macrocode}
%\end{macro}
%
%\begin{macro}{\ProcessPgfOptions}
% The user macro for processing options may have a star.  In that
% case, \cs{@currname} is used for setting keys.
%    \begin{macrocode}
\newcommand*{\ProcessPgfOptions}{%
  \@ifstar
    {\begingroup
       \edef\@tempa{%
         \endgroup
         \noexpand\pgfopts@ProcessOptions
           {/\@currname}}%
     \@tempa}%
    {\pgfopts@ProcessOptions}}
\@onlypreamble\ProcessPgfOptions
%    \end{macrocode}
%\end{macro}
%\begin{macro}{\pgfopts@temp}
% A private storage area is needed by the next macro.
%    \begin{macrocode}
\newcommand*{\pgfopts@temp}{}
%    \end{macrocode}
%\end{macro}
%\begin{macro}{\pgfopts@ProcessOptions}
%  \darg{path}
% The business end of the package.  This is a modified copy of the
% macro \cs{KVO@ProcessOptions} from \pkg{kvoptions}.  Some
% simplifications are made due to the differences in key system, and
% as patching the kernel is ignored.
%    \begin{macrocode}
\newcommand*{\pgfopts@ProcessOptions}[1]{%
  \def\pgfopts@temp{,}%
  \ifx\@currext\@clsextension\else
    \ifx\@classoptionslist\relax\else
%    \end{macrocode}
%\begin{macro}{\pgfopts@CurrentOption}
% For packages, the list of global (class) options needs to be
% checked. The assumption is made that only \emph{explicitly}
% declared keys will be accepted as options here. The strange
% argument to \cs{pgfkeysifdefined} is needed as checking for the key
% itself only works if it has already been used.
%    \begin{macrocode}
      \@for\pgfopts@CurrentOption:=\@classoptionslist\do{%
        \pgfkeysifdefined{#1/\expandafter\pgfopts@sepkeyval%
          \pgfopts@CurrentOption=\@nil/.@cmd}
          {\edef\pgfopts@temp{%
             \pgfopts@temp\pgfopts@CurrentOption,}%
           \@expandtwoargs\@removeelement\pgfopts@CurrentOption
             \@unusedoptionlist\@unusedoptionlist}%
          {}}%
    \fi
  \fi
%    \end{macrocode}
%\end{macro}
% Now the local options have to be processed.  If there are none,
% then the global list is simply copied.  Otherwise, the local
% options are first added to the ``to do'' list stored in \cs{toks@}.
%    \begin{macrocode}
  \begingroup
    \@ifundefined{opt@\@currname.\@currext}
      {\toks@\expandafter{\pgfopts@temp}}
      {\toks@\expandafter\expandafter\expandafter{%
         \csname opt@\@currname.\@currext\endcsname}%
%    \end{macrocode}
% There is work to do if processing options of a class; unknown options
% from the global list have to be completely ignored.
%    \begin{macrocode}
       \ifx\@currext\@clsextension
         \edef\@tempa{\the\toks@}%
         \toks@\expandafter{\pgfopts@temp}%
         \@for\CurrentOption:=\@tempa\do{%
           \pgfkeysifdefined{#1/\expandafter%
             \pgfopts@sepkeyval\CurrentOption=\@nil/.@cmd}
             {\toks@\expandafter{\the\expandafter\toks@
                \expandafter,\CurrentOption}}
             {\ifx\@empty\@unusedoptionlist\@empty
                \global\let\@unusedoptionlist\CurrentOption
              \else
                \expandafter\expandafter\expandafter\gdef
                \expandafter\expandafter\expandafter
                \@unusedoptionlist\expandafter\expandafter
                \expandafter{%
                  \expandafter\@unusedoptionlist
                  \expandafter,\CurrentOption}%
              \fi}}%
%    \end{macrocode}
% For packages, the two lists are simply combined.
%    \begin{macrocode}
      \else
        \toks@\expandafter\expandafter\expandafter{%
            \expandafter\pgfopts@temp\the\toks@}%
      \fi}%
%    \end{macrocode}
% With everything done, the keys are sent to \pkg{pgfkeys} for
% processing.  The default key system in \pkg{pgfkeys} is very
% different from that in \pkg{kvoptions}, and so the system here is
% much less complex.
%    \begin{macrocode}
    \edef\pgfopts@temp{\endgroup
      \noexpand\pgfkeys{#1/.cd,\the\toks@}}%
  \pgfopts@temp
  \let\CurrentOption\@empty
  \AtEndOfPackage{\let\@unprocessedoptions\relax}}
%    \end{macrocode}
%\end{macro}
%\begin{macro}{\pgfopts@sepkeyval}
% A very simply macro to separate a key from a potential value.
%    \begin{macrocode}
\newcommand*{\pgfopts@sepkeyval}{}
\def\pgfopts@sepkeyval#1=#2\@nil{#1}
%    \end{macrocode}
%\end{macro}
% Tidying up.
%    \begin{macrocode}
\pgfopts@catcodes
%    \end{macrocode}
%
%\Finale
%
%\iffalse
%</package>
%<*jawltxdoc>
\NeedsTeXFormat{LaTeX2e}
\ProvidesPackage{jawltxdoc}
\usepackage[T1]{fontenc}
\usepackage{lmodern}
\usepackage[final]{listings,graphicx,microtype}
\usepackage[scaled=0.95]{helvet}
\usepackage[version=3]{mhchem}
\usepackage[osf]{mathpazo}
\usepackage{booktabs,array,url,courier,xspace,etoolbox,varioref}
\usepackage{upgreek,ifpdf,float,caption,longtable,babel}
\floatstyle{plaintop}
\restylefloat{table}
\labelformat{figure}{\figurename~#1}
\labelformat{table}{\tablename~#1}
\ifpdf
  \usepackage{embedfile}
  \embedfile[%
    stringmethod=escape,%
    mimetype=plain/text,%
    desc={LaTeX docstrip source archive for package `\jobname'}%
    ]{\jobname.dtx}
\fi
\IfFileExists{\jobname.sty}
  {\usepackage{\jobname}}{}
\usepackage[numbered]{hypdoc}
\setcounter{IndexColumns}{2}
\newlength\LaTeXwidth
\newlength\LaTeXoutdent
\newlength\LaTeXgap
\setlength\LaTeXgap{1em}
\setlength\LaTeXoutdent{-0.15\textwidth}
\newbox\lst@samplebox
\edef\LaTeXexamplefile{\jobname.tmp}
\lst@RequireAspects{writefile}
\lstnewenvironment{LaTeXexample}[1][example]{%
  \global\let\lst@intname\@empty
  \ifcsname LaTeXcode#1\endcsname
    \expandafter\let\expandafter\LaTeXcode
      \csname LaTeXcode#1\endcsname
    \expandafter\let\expandafter\LaTeXcodeend
      \csname LaTeXcode#1end\endcsname
  \else
    \PackageError{jawltxdoc}
      {Undefined example type `#1'}
      \@ehd
    \let\LaTeXcode\relax
    \let\LaTeXcodeend\relax
  \fi
  \LaTeXcode}
  {\lst@EndWriteFile
   \LaTeXcodeend}
\newcommand*{\LaTeXcodeexample}{%
  \setbox\lst@samplebox=\hbox\bgroup
  \LaTeXcodefloat}
\let\LaTeXcoderesultonly\LaTeXcodeexample
\newcommand*{\LaTeXcodeexampleend}{%
  \egroup
  \setlength\LaTeXwidth{\wd\lst@samplebox}%
  \begin{list}{}{%
    \setlength\itemindent{0pt}
    \setlength\leftmargin\LaTeXoutdent
    \setlength\rightmargin{0pt}}%
    \item
      \setlength\LaTeXoutdent{-0.15\textwidth}
      \begin{minipage}[c]{%
        \textwidth-\LaTeXwidth-\LaTeXoutdent-\LaTeXgap}
        \LaTeXcodefloatend
      \end{minipage}%
      \hfill
      \begin{minipage}[c]{\LaTeXwidth}%
        \hbox to\linewidth{\box\lst@samplebox\hss}%
      \end{minipage}%
  \end{list}}
\let\LaTeXcodenoexample\LaTeXcodeexample
\newcommand*{\LaTeXcodenoexampleend}{%
  \egroup
  \setlength\LaTeXwidth{\wd\lst@samplebox}%
  \begin{list}{}{%
    \setlength\itemindent{0pt}
    \setlength\leftmargin{0pt}
    \setlength\rightmargin{0pt}}%
    \item
      \begin{minipage}[c]{\LaTeXwidth}
        \hbox to\linewidth{\box\lst@samplebox\hss}%
      \end{minipage}%
  \end{list}}
\newcommand*{\LaTeXcodefloat}{%
  \setkeys{lst}{tabsize=4,gobble=3,breakindent=0pt,
    basicstyle=\small\ttfamily,basewidth=0.51em,
    keywordstyle=\color{blue}}%
  \lst@BeginAlsoWriteFile{\LaTeXexamplefile}}
\newcommand*{\LaTeXcodefloatend}{%
  \MakePercentComment\catcode`\^^M=10\relax
  \small
  {\setkeys{lst}{SelectCharTable=\lst@ReplaceInput{\^\^I}%
    {\lst@ProcessTabulator}}%
    \leavevmode \input{\LaTeXexamplefile}}%
  \MakePercentIgnore}
\newcommand*{\LaTeXcoderesultonlyend}{\egroup\LaTeXcodefloatend}
\lstnewenvironment{BibTeXexample}{%
  \global\let\lst@intname\@empty
  \setbox\lst@samplebox=\hbox\bgroup
  \setkeys{lst}{tabsize=4,gobble=3,breakindent=0pt,
    basicstyle=\small\ttfamily,basewidth=0.51em,
    keywordstyle=\color{black}}
  \lst@BeginAlsoWriteFile{\LaTeXexamplefile}}
 {\lst@EndWriteFile
   \LaTeXcodeexampleend}
\newcommand*{\DescribeOption}{%
  \leavevmode\@bsphack\begingroup\MakePrivateLetters
  \Describe@Option}
\newcommand*{\Describe@Option}[1]{\endgroup
              \marginpar{\raggedleft\PrintDescribeEnv{#1}}%
              \SpecialOptionIndex{#1}\@esphack\ignorespaces}
\newcommand*{\SpecialOptionIndex}[1]{\@bsphack
    \index{#1\actualchar{\protect\ttfamily#1}
           (option)\encapchar usage}%
    \index{options:\levelchar#1\actualchar{\protect\ttfamily#1}%
      \encapchar usage}\@esphack}
\newcommand*{\indexopt}[1]{\DescribeOption{#1}\opt{#1}}
\newcommand*{\DescribeOptionInfo}[2]{%
  \DescribeOption{#1}%
  \opt{#1=\meta{#2}}\xspace}
\newcommand*{\ofixarg}[1]{%
  {\ttfamily[}%
  \ifmmode \expandafter \nfss@text \fi
  {%
    \meta@font@select
    \edef\meta@hyphen@restore{%
      \hyphenchar\the\font\the\hyphenchar\font}%
    \hyphenchar\font\m@ne
    \language\l@nohyphenation
    #1\/%
    \meta@hyphen@restore
    }%
    {\ttfamily]}}
\newcommand*{\pkg}[1]{\textsf{#1}}
\newcommand*{\currpkg}{\pkg{\jobname}\xspace}
\newcommand*{\opt}[1]{\texttt{#1}}
\newcommand*{\defaultopt}[1]{\opt{\textbf{#1}}}
\newcommand*{\file}[1]{\texttt{#1}}
\newcommand*{\ext}[1]{\file{.#1}}
\newcommand*{\latin}[1]{\emph{#1}}
\newcommand*{\etc}{%
  \@ifnextchar.
    {\latin{etc}}
    {\latin{etc}.\xspace}}
\newcommand*{\eg}{%
  \@ifnextchar.
    {\latin{e.g}}
    {\latin{e.g}.\xspace}}
\newcommand*{\ie}{%
  \@ifnextchar.
    {\latin{i.e}}
    {\latin{i.e}.\xspace}}
\newcommand*{\etal}{%
  \@ifnextchar.
    {\latin{et~al.}}
    {\latin{et~al}.\xspace}}
\newcommand*{\AMS}{{\protect\usefont{OMS}{cmsy}{m}{n}%
  A\kern-.1667em\lower.5ex\hbox{M}\kern-.125emS}}
\providecommand*{\eTeX}{\ensuremath{\varepsilon}-\TeX}
\DeclareRobustCommand*{\XeTeX}
  {X\kern-.125em\lower.5ex\hbox{\reflectbox{E}}\kern-.1667em\TeX}
\providecommand*{\CTAN}{\textsc{ctan}}
\patchcmd{\@addmarginpar}
  {\@latex@warning@no@line {Marginpar on page \thepage\space moved}}
  {\relax}{}{}
\newcounter{argument}
\g@addto@macro\endmacro{\setcounter{argument}{0}}
\newcommand*\darg[1]{%
  \stepcounter{argument}%
  {\ttfamily\char`\#\theargument~:~}#1\par\noindent\ignorespaces}
\newcommand*\doarg[1]{%
  \stepcounter{argument}%
  {\ttfamily\makebox[0pt][r]{[}%
   \char`\#\theargument]:~}#1\par\noindent\ignorespaces}
%</jawltxdoc>
%\fi
